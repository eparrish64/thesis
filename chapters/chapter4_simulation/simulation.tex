\chapter{Simulation}\label{chap:sim}
	The data collected by the \gls{ATLAS} experiment must be compared to a well understood dataset. This dataset is most often a dataset of simulation particle collisions that approximate to great precision physics processes and particle interaction with detector material, as well as the detector's response. Figure \ref{fig:simulation} shows the chain of simulations by which these datasets are produced.
	\begin{figure}[!ht]
	\centering
	\includegraphics[width=.95\textwidth,keepaspectratio=true]{chapters/chapter4_simulation/images/Simulation_Chain.png}
	\caption{\label{fig:simulation} A pictorial representation of the simulation chain used in the \gls{ATLAS} experiment \cite{Wanotayaroj:2242196}.}
	\end{figure}
	
	Many particle physics experiments, \gls{ATLAS} included, use \gls{MC} simulation techniques to produce these datasets. Monte Carlo simulation techniques use repeated random sampling of underlying probability density functions to closely model various processes. 

	\section{Event Generation and Hadronization}\label{sec:event-gen}
		Since protons and other hadrons are not fundamental particles, it is impossible to know the exact constituents (partons) that interacted during a collision. To mimic this intrinsic probabilistic nature, \acrfullpl{PDF} are used. A \gls{PDF} models the probability of any parton within a proton (or hadron) to carry a fraction of the beam energy at a given hadron momentum. The \gls{PDF} and subsequent inelastic hard scattering of the interacting partons are modeled via a \gls{ME} calculation, which can be depicted through Feynman diagrams. This \gls{ME} calculation is done to fixed order in perturbation theory, \gls{LO}, \gls{NLO}, \gls{LL}, etc. This first level event generation can be done by a myriad of \gls{MC} event generators. Often specific choices are made based on individual generator performance for a given physics process.

		The next step in the simulation chain is the parton showering and hadronization. This can be done with a different set of \gls{MC} simulations. Parton showering and hadronization are complex, computationally expensive steps to simulate and are done iteratively. An example of a parton shower generator output can be seen in Figure \ref{fig:hadronization}.

		The \gls{MC} generators used in this dissertation are Pythia \cite{pythia}, Powheg-Box \cites{powheg-1}{powheg-2}, and Sherpa \cite{sherpa}.

		\begin{figure}[!ht]
		\centering
		\includegraphics[width=\textwidth,keepaspectratio=true]{chapters/chapter4_simulation/images/tth_hadronization_gen.png}
		\caption{\label{fig:hadronization} A pictorial representation of a parton shower of a t$\bar{\mathrm{t}}$H event \cite{Wanotayaroj:2242196}.}
		\end{figure}	

	\section{Detector Simulation}\label{sec:detector-sim}
		The final step in the simulation chain is simulating the particle's interaction with the detector material and the detector's response. Up until this point, the \gls{MC} generators used are generic non-experiment dependent simulations. The \gls{ATLAS} collaboration uses a GEANT4 based generator suite to simulate these interactions \cite{GEANT4}. These detailed simulations include all support structure, material densities, readout electronics, and digitization in order to fully simulate the path of a real particle through the \gls{ATLAS} detector. In fact, these simulations are often too detailed to produce enough statistics for physics analyses. In the full simulation, around $80\%$ of the simulation time is spent on particles traversing the calorimeters and $75\%$ is spent on electromagnetic particles alone \cite{ATLAS-simulation}. Instead, several methods were developed to speed up the simulation known as FAST simulations. A detailed description of the \gls{ATLAS} simulation chain and options can be seen in \cite{ATLAS-simulation}. The final simulated dataset is output into a raw data format identical to real data coming off of the \gls{ATLAS} detector. 

		