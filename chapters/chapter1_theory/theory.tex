\chapter{Theory}
    In this chapter, the theoretical motivation of a search for \HpLong is described. Firstly, a review of the Standard Model of particle physics (SM) is laid out, then a brief overview of Supersymmetry focusing on the Minimal Supersymmetric Standard Model (MSSM) Higgs sector is detailed. Finally, the Type II 2-Higgs Doublet Model's (2HDM) relation to the \Hp production cross section and subsequent branching ratio into SM particles is described as motivation for the choice of studying \HpLong.

\section{The Standard Model}
	The Standard Model of particle physics is a mathematical model that describes all known matter and forces. The Standard Model is built upon a gauge group of type $SU(3)_C \times SU(2)_L \times U(1)_Y$. The $SU(3)_C$ term dictates the strong interaction while the $SU(2)_L \times U(1)_Y$ term describes the electroweak interaction. These interactions occur between fundamental particles called fermions that comprise the known matter of the universe. The interactions, or forces, are mediated by fundamental particles called bosons. 

	\subsection{Particles}        
		The particles that make up the Standard Model are separated into two groups according to their intrinsic angular momentum charge, or spin. Fermions are those that carry half-integer spin while Bosons carry full integer spin values.

		\textcolor{red}{Generations??}

		\textcolor{red}{Include standard model diagram}

		\subsubsection{Fermions}	
		Fermions are even further divided into two groups, quarks and leptons. The quarks participate in the strong interaction via their color charge. Quarks cannot exist as a singular particle and thus combine into hadrons. Leptons carry no color charge and therefore do not participate in strong force interactions. The fermions in the standard model all participate in the electroweak interaction. However, the electromagnetic interaction is limited to those fermions that carry an electromagnetic charge.

		Fermions can then be split further into three generations

		\subsubsection{Bosons}
			\textcolor{red}{NEEDS TO BE DONE}

	\subsection{Interactions}
		\textcolor{red}{NEEDS TO BE DONE}

		\subsubsection{Electromagnetic Interaction}

			\textcolor{red}{NEEDS TO BE DONE}

		\subsubsection{Weak Interaction}
			\textcolor{red}{NEEDS TO BE DONE}

		\subsubsection{Strong Interaction}
			\textcolor{red}{NEEDS TO BE DONE}

	\subsection{The Higgs Mechanism}
		\textcolor{red}{NEEDS TO BE DONE}

\section{Supersymmetry}
	\textcolor{red}{NEEDS TO BE DONE}

	\subsection{MSMM Particles}
		\textcolor{red}{NEEDS TO BE DONE}

	\subsection{R-Parity}
		\textcolor{red}{NEEDS TO BE DONE}

	\subsection{The MSSM Higgs Sector}
		\textcolor{red}{NEEDS TO BE DONE}

\section{Charged Higgs Bosons}
	\textcolor{red}{NEEDS TO BE DONE}

	\subsection{Previous Result}
		\textcolor{red}{NEEDS TO BE DONE}