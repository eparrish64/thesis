\chapter{Conclusion}

A long-term goal of the \gls{LHC} physics program is understanding Electroweak Symmetry Breaking. Measurement of the Higgs trilinear coupling probes the nature of the Higgs potential and provides a precision test of electroweak theory. Studying di-Higgs production provides a direct handle on the Higgs trilinear coupling, and the \yybb decay channel is an interesting channel due to the high \Hbb branching ratio as well as the excellent $H\rightarrow \yy$ mass resolution and trigger efficiency.

A search for resonant and non-resonant di-Higgs production in the \yybb final state has been presented. This search was performed using 36.1 \ifb of $\sqs = \unit{13}{\TeV}$ data from $pp$ collisions collected by the ATLAS detector in 2015-2016 data, the first two years of \RunTwo. The non-resonant search set limits on the $\hh\rightarrow\yybb$ \xsec times branching ratio, with an upper observed (expected) limit of 0.73 (0.93) pb. The observed (expected) 95\% \gls{CL} constraint on the Higgs boson trilinear coupling is set between $-8.2 < \klambda < 13.3$ ($-8.5 < \klambda < 13.7$). A model-independent resonant search has been presented, setting limits on a generic scalar resonance between 260 and 1000 GeV under the narrow-width approximation. These observed (expected) limits range from 0.85 (0.92) pb at the lowest mass hypothesis to 0.13 (0.15) pb to the highest mass hypothesis.

This analysis is sensitive to the square of the photon identification efficiency, and in the future of this search, improvements to photon identification can allow a loosening of the signal selection, which can help the overall significance. By using topological cluster moments as inputs into the current cut-based model, an improvement of as much as 12\% in background rejection for the same signal efficiency is found. Replacing the current cut-based model with a \pt-inclusively trained \gls{BDT} leads to an improvement of 22\% in background rejection. By combining both optimization strategies, an improvement of 27\% is shown.

Direct improvements to the analysis have also been presented, studying the \gls{VBF} production mode. This provides sensitivity to different couplings compared to prior \gls{ggF}-only searches, notably the interaction between two Higgs bosons and two vector bosons. Additionally, constraints on di-Higgs production can be improved by incorporating a signal region targeting \glsfirst{VBF} production. This signal region is defined through the use of a multiclass \gls{BDT} with a dedicated class to differentiate the production mode from ggF HH production, as well as classes to differentiate from the dominant \yy-continuum background, and $ttH$ mono-Higgs background. Through defining this \gls{VBF} enriched signal region, a 9.7\% improvement in Asimov significance is achieved.