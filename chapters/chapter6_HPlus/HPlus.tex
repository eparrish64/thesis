\chapter{Search for Charged Higgs Bosons}\label{chap:hpana}
	This chapter details a search for a charged Higgs boson decaying to a hadronically decaying tau lepton and a neutrino; the phenomenology is discussed in Section \ref{sec:Hpm}. The search contains two subchannels, \taujets and \taulep based on the decay of the associated top quark in the collision event. The \taujets subchannel ($t\to Wb, \, W \to q\bar{q}$)  has a higher branching fraction, leading to higher sensitivity at larger \mHpm values. The \taulep subchannel ($t\to Wb, \, W \to \ell \nu$)  has a much lower branching fraction, but takes advantage of single-lepton triggers which enhance background suppression of \acrshort{QCD} $\mathrm{jet} \, \to \, \tau$ fakes. This leads to an increased sensitivity at lower \mHpm values. The extra neutrino in the \taulep decay mode creates extra difficulties in separating signal from background in this subchannel by adding a significant contribution to the \Etm calculation for the event. 

	The search described by this dissertation uses a profile likelihood ratio as the test statistic in a simultaneous fit in two \glspl{CR} and three \glspl{SR}. The discriminating variable is chosen to be the output score distribution of a \gls{MVA}. In the previous publication described in Section \ref{ssec:Prev Hpm} several \gls{BDT} were used, binned in \mHpm; this analysis uses a \gls{PNN} to classify events as signal-like or background-like.

	This chapter discusses in detail the entire analysis, including the signal signatures, event selections, analyzed datasets, modeling of backgrounds, training and evaluation of classifiers, studies of systematic uncertainties, and results.

	\section{Signature and Event Selection}\label{sec:signal}
		As shown in Figure \ref{fig:hpm-diagrams}, the production of the \Hpm is dependent on its mass \mHpm. Table \ref{tab:hplus-production} shows the production mechanisms for \mHpm values in bins of the top quark mass $m_t$ as well as the main decay mode (and theoretical constraints), and the main source of background. Three mass ranges are defined, low mass $80 \leq \mHpm \leq 130 $ GeV, intermediate mass $140 \leq \mHpm \leq 190$, and high mass $200 \leq \mHpm \leq 3000$ GeV. In the low mass bin the \Hpm takes the place of the $W^{\pm}$ in the top quark decay. Results of searches for other non-standard decays of top quarks limits on the branching ratio can be seen in Reference \cite{pdg}. The two subchannels of this analysis have similar signal signatures with a hard-scatter source of \Etm, one \tauhad, and at least 1 \bjet from the associated top decay. In the \taulep subchannel there is an extra requirement of a lepton (e or $\mu$). Due to the variable amount of energy available to the final state products based on \mHpm the event topology changes as a function of \mHpm. As described in Section \ref{sec:mva}, classifiers are trained and evaluated in \mHpm bins to account for the varying event topology.

		\begin{table}[!thp]
			\centering
			\resizebox{\textwidth}{!}{
			\begin{tabular}{| c | c | c | c |}
			\hline
			\textbf{\Hpm Mass} & \textbf{Production Mechanism} & \textbf{Decay}  & \textbf{Main Backgrounds}\\
			\hline \hline
			\multicolumn{1}{|c|}{\multirow{2}{*}{$\mHpm < m_{t}$}} 		& \begin{tabular}[c]{@{}c@{}} double-resonant $t \to H^\pm b$ (LO) \\ \includegraphics[width=.19\textwidth]{chapters/chapter6_HPlus/images/double_resonant_production_low_mass.png} \end{tabular} 											& \begin{tabular}[c]{@{}c@{}} \HpmLong \\ (low $\tanb \implies$ $\Hpm \to cs$ or $\Hpm \to cb$ ) \end{tabular} 																	& \ttbar, single-top \\ \hline

			\multicolumn{1}{|c|}{\multirow{3}{*}{$\mHpm \simeq m_{t}$}} & \begin{tabular}[c]{@{}c@{}} non-resonant $t \to \Hpm b$ (LO) \\ \includegraphics[width=.19\textwidth]{chapters/chapter6_HPlus/images/non_resonant_production_intermediate_mass.png} \\ interferences taken into account \end{tabular} 	& $\Hpm \to \tau \nu$  																																									& \ttbar, single-top \\ \hline

			\multicolumn{1}{|c|}{\multirow{3}{*}{$\mHpm > m_{t}$}}		& \begin{tabular}[c]{@{}c@{}} single-resonant $gg \to tbH^\pm$ (NLO) \\ \includegraphics[width=.19\textwidth]{chapters/chapter6_HPlus/images/single_resonant_production_large_mass.png} \end{tabular} 										& \begin{tabular}[c]{@{}c@{}} $\Hpm \to tb$ \\ ($cos(\beta-\alpha) \simeq 0$ and large $tan(\beta) \implies \Hpm \to \tau \nu$ \\ $BR(\HpmLong) \simeq 10-15\%$ ) \end{tabular} & multi-jet \\ \hline

			\end{tabular}}
			\caption{\Hpm production mechanisms based on \mHpm, dominant \Hpm decay mode, and the main backgrounds associated with the diagram.}
			\label{tab:hplus-production}
		\end{table}

		\subsection{Object Definitions}\label{ssec:object-def}
			After physics objects are reconstructed additional kinematic and identification cuts are applied to allow for high identification efficiency while keeping significant statistics \cites{tau-id-rnn}{jet-calibration}{b-tagging}{electron-perf}{muon-id}. Table \ref{tab:object-defs} shows the identification requirements on all objects used in the analysis. In both subchannels \tauhad candidates are required to fit the medium working point described in Section \ref{ssec:reco-tau} that corresponds to a 75\% efficiency for 1-prong and 60\% efficiency for 3-prong \tauhad identification, an $\abs{\eta}$ cut of $< 2.3$ that also excludes the gap and crack region of the ATLAS calorimeters at $1.37 < \abs{\eta} < 1.52$, and an overlap removal with electrons. For the \taujets subchannel, the \tauhad \pt is required to be greater than 40 GeV and greater than 30 GeV for the \taulep subchannel. Although muons and electrons are not part of the \taujets signal final state, a loose identification and isolation requirement is used to veto events; while the \taulep subchannel requires there to be either an electron or a muon that passes the tight identification and isolation requirements as well as a \pt above 30 GeV. The jets in candidate events are required to have greater than 25 GeV in \pt and are made with the anti-$k_t$ algorithm with R=0.4. Any event that contains a jet with $\pt > 25$ GeV and fails quality cuts is discarded. This ensures that no jet is consistent with having originated from instrumental effects or non-collision backgrounds. Jets tagged as \bjets are done so at a 70\% efficient working point using the DL1r tagger described in Section \ref{ssec:flavor-tagging}. The \pt requirements listed above for each physics object were chosen to maximize identification efficiency of each object. An example of identification efficiency can be seen for electrons in Figure \ref{fig:electron-efficiency}. The chosen cut of $\pt^{e}>30$ GeV corresponds to an identification efficiency of $\approx 70\%$. Similar choices for $\tau$ leptons \cite{tau-id-rnn}, jets \cite{jet-calibration}, \bjets \cite{b-tagging}, \Etm \cite{met-perf}, and muons \cite{muon-calibration} were made. 

			\begin{table}[!thp]
				\centering
				\resizebox{\textwidth}{!}{
				\begin{tabular}{| c | l | l |}
				\hline
				Object & \textbf{\taujets} & \textbf{\taulep} \\
				\hline \hline
				\multicolumn{1}{|c|}{\multirow{3}{*}{\tauhad}} & \begin{tabular}[c]{@{}c@{}}Leading reconstructed $\tau$ (regardless of its ID), \\ mediumID$^{*}$, $\pt > 40$ GeV, $\abs{\eta}^{***} < 2.3$, $e$ OLR \end{tabular} & \begin{tabular}[c]{@{}c@{}} Leading reconstructed $\tau$ (regardless of its ID), \\ mediumID$^{*}$, $\pt > 30$ GeV, $\abs{\eta}^{***} < 2.3$, $e$ OLR \end{tabular} \\[4ex] \hline
				\multicolumn{1}{|c|}{\multirow{3}{*}{$e$}} & \begin{tabular}[c]{@{}c@{}} LoseLLH, $\pt > 20$ GeV, $\abs{\eta}^{***} < 2.47$, \\ Loose isolation, IP cuts \end{tabular} &  \begin{tabular}[c]{@{}c@{}} TightLLH, $\pt > 30$ GeV, $\abs{\eta}^{***} < 2.47$, \\ Tight isolation, IP cuts \end{tabular} \\[4ex] \hline
				\multicolumn{1}{|c|}{\multirow{3}{*}{$\mu$}} & \begin{tabular}[c]{@{}c@{}} LooseID, $\pt > 20$ GeV, $\abs{\eta} < 2.5$, \\Loose isolation, IP cuts \end{tabular} & \begin{tabular}[c]{@{}c@{}} TightID, $\pt > 30$ GeV, $\abs{\eta} < 2.5$,\\ Tight isolation, IP cuts \end{tabular} \\[4ex] \hline 
				\multicolumn{1}{|c|}{\multirow{3}{*}{jet}} & \begin{tabular}[c]{@{}c@{}} AntiKt4EMPFlow, $\pt > 25$, GeV $\abs{\eta} < 2.5$,\\ JVT$^{**}$  $> 0.59$, Btag=70\%, DL1r \end{tabular} & \begin{tabular}[c]{@{}c@{}} AntiKt4EMPFlow, $\pt > 25$ GeV, $\abs{\eta} <2.5$, \\ JVT$^{**}$  $ > 0.59$ , Btag=70\%, DL1r \end{tabular} \\[4ex] \hline
				\end{tabular}}
				\caption{Definitions of physics objects used in this analysis.}
				\label{tab:object-defs}
			\end{table}

			\begin{figure}
			\begin{center}
			\includegraphics[width=\textwidth,keepaspectratio=true]{chapters/chapter6_HPlus/images/Electron_ID_Efficiency.png}
			\end{center}
			\caption{The electron identification efficiency in $Z\to ee$ events in data as a function of $E_{T}$ for the Loose, Medium, and Tight operating points. The efficiencies are obtained by applying data-to-simulation efficiency ratios measured in $J/\Psi \to ee$ and $Z \to ee$ events to $Z \to ee$ simulation. The inner uncertainties are statistical and the total uncertainties are the statistical and systematic uncertainties in the data-to-simulation efficiency ratio added in quadrature. The bottom panel shows the data-to-simulation ratio. Taken from Reference \cite{electron-perf}. }
			\label{fig:electron-efficiency}
			\end{figure}

			% \begin{columns}
			% \column{.3\textwidth}
			% \begin{itemize}
			%   \footnotesize
			%   \item $\tau$ mediumID$^{*}$
			%   \begin{itemize}
			%     \tiny
			%     \item 1-prong: 75\% ID eff 
			%     \item 3-prong: 60\% ID eff
			%   \end{itemize}
			% \end{itemize}
			% \column{.3\textwidth}
			% \begin{itemize}
			%   \footnotesize
			%   \item JVT$^{**}$
			%     \begin{itemize}
			%       \tiny
			%       \item $\pt < 60$ GeV
			%       \item $\abs{\eta}<2.4$
			%   \end{itemize}
			% \end{itemize}
			% \column{.3\textwidth}
			% \begin{itemize}
			%   \footnotesize
			%   \item $\abs{\eta}^{***}$
			%     \begin{itemize}
			%       \tiny
			%       \item $1.37 < \abs{eta} < 1.52 $ excluded
			%   \end{itemize}
			% \end{itemize}
			% \end{columns}

		\subsection{Event Selections}\label{ssec:event-selection}
			Each subchannel signal region has stricter requirements than the object definitions described in Section \ref{ssec:object-def}. Table \ref{tab:signal-regions} details these selections. The channels differ in the triggers used; the \taujets subchannel relies on \Etm triggers while the \taulep subchannel relies on single lepton triggers. Detailed information on the triggers and the efficiencies of the trigger options can be found in References \cite{MET-Trigger} (\Etm), \cite{Electron-Trigger} ($e$), and \cite{Muon-Trigger} ($\mu$).  Due to the difficulty of separating signal from background and the large amount of background, the \taujets subchannel has a higher \pt cut on the \tauhad of $40$ GeV as opposed to the \taulep value of $30$ GeV. In addition, a higher value of \Etm of $150$ GeV is required for the \taujets subchannel compared to $50$ GeV in the \taulep subchannel. In the \taujets subchannel a value of $50$ GeV is also required of the transverse mass $m_{T}$ defined as 
			\begin{equation}\label{eqn:transverse-mass}
			m_{T} = \sqrt{ 2 \pt^{\tau} \Etm (1 - cos \Delta \phi_{\tau,\Etm}) }
			\end{equation}
			The \taulep has no such requirement, but does require the \tauhad and lepton to have opposite electromagnetic charge. A set of orthogonal \glspl{CR} are defined for each subchannel to verify proper background modelling and are described in Section \ref{sec:bkg-modeling}. The acceptance of signal in the signal regions defined in Table \ref{tab:signal-regions} is shown in Figure \ref{fig:signal-acceptance}. The instability of the data points is due to the boundaries of the \Hpm production diagrams.  Due to the larger branching fraction of $t \to W+b;  W \to q\bar{q}$ as opposed to $t \to W+b; W \to \ell + \nu$ the \taujets subchannel has a factor of 10 larger signal acceptance than the \taulep subchannel. In both channels, the signal acceptance decreases for \mHpm values $> 1000$ GeV. This is an artifact of objects becoming boosted, meaning their decay products are extremely collimated, resulting in lower efficiencies for object identification.

			\begin{table}[!thp]
				\centering
				\resizebox{.75\textwidth}{!}{
				\begin{tabular}{| c | c |}
				\hline
				\textbf{$\tau + jets$ SR } & \textbf{$\tau + \ell$ SR} \\
				\hline \hline
				$\Etm$ Trigger  																& Single lepton triggers (e or $\mu$) \\[1.2ex] \hline
				1 \tauhad ; $\pt^{\tau} > 40$ GeV 							& 1 \tauhad ; $\pt^{\tau} > 30$ GeV\\[1.2ex] \hline
				0 $\ell$ (e or $\mu$) ; $\pt^{\ell} > 20$ GeV  	& 1 $\ell$ (e or $\mu$) ; $\pt^{\ell} > 30$ GeV \\[1.2ex] \hline
				$\geq$ 3 jets ; $\pt^{j} > 25$ GeV  						& $\geq$ 1 jet ; $\pt^{j} > 25$ GeV \\[1.2ex] \hline
				$\geq$ 1 \bjets ; $\pt^{\bjet} > 25$ GeV 				& $\geq$ 1 \bjets ; $\pt^{\bjet} > 25$ GeV \\[1.2ex] \hline
				\Etm$ > 150$ GeV 																& \Etm$ > 50$ GeV \\[1.2ex] \hline
				$m_{T}(\tau,\Etm) > 50$ GeV 										& Opposite sign $\tau$ and $\ell$ \\[1.2ex] 
				\hline
				\end{tabular}}
				\caption{\taujets and \taulep signal region definitions.}
				\label{tab:signal-regions}
			\end{table}

			\begin{figure}[!ht]
				\centering
				\subfloat[\label{fig:signal-acceptance_a}]{\includegraphics[width=0.5\textwidth]{chapters/chapter6_HPlus/images/Signal_Acceptance_Efficiency_SR_TAUJET.pdf}}
				\subfloat[\label{fig:signal-acceptance_b}]{\includegraphics[width=0.5\textwidth]{chapters/chapter6_HPlus/images/Signal_Acceptance_Efficiency_SR_TAULEP.pdf}}
				\caption{\label{fig:signal-acceptance} Signal acceptance as a function of the charged Higgs boson mass for both the \taujets (a) and \taulep subchannels (b). Statistical errors are shown but are negligible.}
			\end{figure}

	\section{Datasets}\label{sec:datasets}
		This analysis uses the full Run-2 ATLAS dataset collected between 2015 and 2018 corresponding to $139.0 \pm 2.4$ \ifb \cite{lumi-run2}. The datasets used are required to be included in the ATLAS ``Good Run Lists'' (GRLs), meaning they have passed nominal data quality checks with all detector subsystems operating within normal conditions. Further event cleaning is applied that removes events in which a reconstructed jet originated from detector noise or non-collision backgrounds. The collection of data throughout Run-2 can be seen in Figure \ref{fig:lhc-lumi}.

		\subsection{Signal Modeling}\label{ssec:sig-modeling}
		\gls{MC} simulations of \Hpm signal events are generated at varying orders dependent on \mHpm. In all cases, the 2HDM Type II model described in Section \ref{ssec:2HDM} is assumed and the generator MadGraph is used. The lower mass range corresponding to $\mHpm < 140$ GeV where a \Hpm takes the place of a $W^{\pm}$ in a top decay is generated at LO. The intermediate mass range of $140 \leq \mHpm < 200 $ GeV is generated at LO, taking into account the non-resonant, single-top resonant and double-resonant diagrams and their interferences. In this mass range, the final state contains one \Hpm, one $W^{\pm}$, and two b quark. For charged Higgs masses of 200 GeV and above, the \Hpm is produced in association with a top quark and is generated at NLO. The Powheg-box v2 \cites{powheg-1}{powheg-2} generator is used with the NNPDF3.0 \gls{NLO} \gls{PDF} \cite{PDFs-2} set in the matrix element calculations to generate \ttbar and single top-quarks in the W t- and s-channels. In all cases, the parton generator is interfaced with Pythia v8.230 \cite{pythia} with the NNPDF2.3 \gls{LO} \gls{PDF} \cite{PDFs-1} using the A14 underlying event tuning parameters \cite{Pythia8-tunes}. Table \ref{tab:signal-generated} shows the cross section and raw number of events generated for each \mHpm point for both subchannels.

		\begin{table}[!thp]
			\centering
			\resizebox{.65\textwidth}{!}{
			\begin{tabular}{| l | l | l | l |}
			\hline
			\mHpm [GeV] 	& $\sigma$ [pb] 	& \taulep Generated Events 	& \taujets Generated Events 	\\ \hline
			80 				& 61.639 			& 220k 						& 110k							\\
			90 				& 52.823 			& 220k 						& 110k							\\
			100 			& 43.777 			& 220k 						& 110k							\\
			110				& 34.770 			& 220k 						& 110k							\\
			120 			& 26.092 			& 220k 						& 110k							\\
			130 			& 18.069 			& 220k 						& 110k							\\ \hline
			140 			& 15.023 			& 220k 						& 220k							\\
			150 			& 7.681 			& 220k 						& 220k							\\
			160 			& 2.665 			& 220k 						& 220k							\\
			170 			& 0.63748 			& 220k 						& 220k							\\
			180 			& 0.52979 			& 220k 						& 220k							\\
			190 			& 0.47201 			& 220k 						& 220k							\\ \hline
			200 			& 0.55632 			& 110k 						& 220k							\\
			225 			& 0.44081 			& 110k 						& 220k							\\
			250 			& 0.3573 				& 110k 						& 220k							\\
			275 			& 0.28592 			& 110k 						& 220k							\\
			300 			& 0.23373 			& 110k 						& 220k							\\
			350 			& 0.15774 			& 110k 						& 220k							\\
			400 			& 0.10818 			& 110k 						& 220k							\\
			500 			& 0.054139 			& 110k 						& 220k							\\
			600 			& 0.02847 			& 110k 						& 220k							\\
			700 			& 0.015764 			& 110k 						& 220k							\\
			800 			& 0.009067 			& 110k 						& 220k							\\
			900 			& 0.005324 			& 110k 						& 220k							\\
			1000 			& 0.003271 			& 110k 						& 220k							\\
			1200 			& 0.001311 			& 110k 						& 220k							\\
			1400 			& 0.000558 			& 110k 						& 220k							\\
			1600 			& 0.000252 			& 110k 						& 220k							\\
			1800 			& 0.000120 			& 110k 						& 220k							\\
			2000 			& 0.0000587 		& 110k 						& 220k							\\
			2500 			& 0.0000111			& 110k 						& 220k							\\
			3000 			& 0.00000234		& 110k 						& 220k							\\ \hline
			\end{tabular}}
			\caption{For each \Hpm mass the generator xcross-section $(\sigma \times BR(\HpmLong))$ is given, as well as the number of generated events for both \taulep and \taujets subchannels.}
			\label{tab:signal-generated}
		\end{table}
		% \pagebreak

	\section{Background Modeling}\label{sec:bkg-modeling}
		The main sources of backgrounds are shown in Table \ref{tab:backgrounds}, separated between backgrounds with a prompt \tauhad in the hard scatter process and those that arise from the misidentification of other physics objects as a \tauhad. The cross section of all simulated background samples and the relevant generators can be seen in Table \ref{tab:bkg-xs}. Control regions that are designed to be orthogonal to the signal region are created for both subchannels in order to study the modeling of the backgrounds. These control regions are defined by the cuts in Table \ref{tab:taujet-control-regions} (\taujets) and Table \ref{tab:taulep-control-regions} (\taulep). For the \taulep subchannel the Same Sign and b-veto control regions are further split into two control regions, one that requires a $\mu$ in the event and another that requires an electron. 

		

    \begin{table}[!thp]
      \centering
      \resizebox{\textwidth}{!}{
      \begin{tabular}{| l | l |}
      \hline
      Backgrounds w/ prompt \tauhad & Backgrounds w/ fake $\tau$ \\
      \hline \hline
      $t\bar{t}$ estimated with \gls{MC}       & Fake $j \to \tau$ estimated with data driven fake factor method \\ \hline
      $W(Z)+jets$ estimated with \gls{MC}         & Fake $\ell \to \tau$ estimated with \gls{MC}, validated on $Z \to ee$\\ \hline
      Diboson estimated with \gls{MC} & \\
      \hline
      \end{tabular}}
      \caption{Dominant backgrounds from prompt \tauhad and fake \tauhad candidates.}
      \label{tab:backgrounds}
    \end{table}

		\begin{table}[!thp]
			\begin{center}
			\small
			\resizebox{0.5\textwidth}{!}{
			\begin{tabular}{|c||c|c|}
			\hline
			Background process & Generator \& & Cross section \\
			  & parton shower & number(s) [pb] \\
			\hline \hline
			$\begin{array}{c}
			$\ttbar$~\mbox{with at least one lepton $\ell$} \\ 
			\end{array}$ &
			$\begin{array}{c}
			\mbox{{\textsc Powheg}}~\& \\
			\mbox{{\textsc Pythia8}}
			\end{array}$
			& 729.77* \\
			\hline
			$\begin{array}{c}
			\mbox{Single top-quark}\\ 
			\mbox{$t$-channel}
			\end{array}$ & & 59.17* \\
			$\begin{array}{c}
			\mbox{Single top-quark}\\ 
			\mbox{$s$-channel}
			\end{array}$ &
			$\begin{array}{c}
			\mbox{{\textsc Powheg}}~\& \\
			\mbox{{\textsc Pythia8}}
			\end{array}$ & 3.29* \\
			$\begin{array}{c}
			\mbox{Single top-quark}\\ 
			\mbox{$Wt$-channel}
			\end{array}$  & & 83.83  \\
			\hline
			$\begin{array}{c}
			~ \\
			W(\ell\nu) + \mbox{jets} \\ 
			~ \\
			\end{array}$ &
			$\begin{array}{c}
			~ \\
			\mbox{Sherpa 2.2.1} \\ 
			~ \\
			\end{array}$ &
			$\begin{array}{c}
			2.0\times 10^4 \\
			2.0\times 10^4 \\ 
			2.0\times 10^4 \\
			\end{array}$ \\
			\hline
			$\begin{array}{c}
			Z/\gamma^{\ast}(\ell\ell,\nu\nu) + \mbox{jets} \\
			\end{array}$ & 
			$\begin{array}{c}
			~ \\
			\mbox{Sherpa 2.2.1} \\ 
			~ \\
			\end{array}$  &
			$\begin{array}{c}
			2.1 \times 10^3  \\
			2.1 \times 10^3  \\ 
			2.1 \times 10^3  \\
			\end{array}$ \\

			\hline
			$WW$  &  & 54.81 \\
			$WZ$  &  $\begin{array}{c}
			\mbox{{\textsc Powheg}}~\& \\
			\mbox{{\textsc Pythia8}}
			\end{array}$ & 16.34 \\
			$ZZ$  &  & 8.94 \\
			\hline
			\end{tabular}}
			\normalsize
			\caption{\label{tab:bkg-xs}
			Cross sections for the main \acrshort{SM} 
			background samples at \sqs. 
			Here, $\ell$ refers to the three lepton families $e$, $\mu$ and 
			$\tau$. All background cross sections are normalized to NNLO predictions, 
			except for diboson events, where the NLO prediction is used. A '*' indicates
			that the quoted cross section for the sample is neglecting leptonic/hadronic
			branching ratios.
			}
			\end{center}
		\end{table}

		% \begin{table}[!thp]
		% 	\begin{subtable}[c]{0.45\textwidth}
		% 		\centering
		% 		\begin{tabular}{| c |}
		% 			\hline
  %       	\textbf{\ttbar Control Region} \\ \hline \hline
  %         1 \tauhad \\
  %         $\pt^{\tau} > 40 $ GeV \\
  %         $\geq 3 $ jets\\
  %         $\geq 2 $ \bjets\\
  %         $\Etm > 150 GeV$ \\
  %         $m_{T}(\tau, \Etm) < 100 \: GeV$ \\
  %         \hline
		% 		\end{tabular}
		% 		\subcaption{\ttbar modeling}
		% 	\end{subtable}
		% 	\begin{subtable}[c]{.45\textwidth}
		% 		\centering
		% 		\begin{tabular}{| c |}
		% 			\hline
		% 			\textbf{b-veto Control Region} \\ \hline \hline
		% 			1 \tauhad \\
		% 			$\pt^{\tau} > 40 \: GeV $  \\
		% 			$\geq 3$ jets\\
		% 			% $n\_bjets > 2$ \\
		% 			$\pt^{jet} > 25 \: GeV$ \\
		% 			$\Etm > 150 GeV$ \\
		% 			$m_{T}(\tau, \Etm) > 50 \: GeV$ \\
		% 			b veto \\
		% 			$\ell$ veto \\
		% 			\hline
		% 		\end{tabular}
		% 		\subcaption{Close to signal region}
		% 	\end{subtable}
		% 	\begin{subtable}[c]{.45\textwidth}
		% 		\centering
		% 		\begin{tabular}{| c |}
		% 			\hline
		% 			\textbf{W+Jets Control Region} \\ \hline \hline
		% 			1 \tauhad \\
		% 			$\pt^{\tau} > 40 \: GeV $  \\
		% 			$\geq 3$ jets \\
		% 			$\pt^{jet} > 25 \: GeV$ \\
		% 			$\Etm > 150 GeV$ \\
		% 			$m_{T}(\tau, \Etm) > 100 \: GeV$ \\ 
		% 			b veto \\
		% 			$\ell$ veto \\
		% 			\hline
		% 		\end{tabular}
		% 		\subcaption{W+Jets modeling}
		% 	\end{subtable}
		% 	\begin{subtable}[c]{.45\textwidth}
		% 		\centering
		% 		\begin{tabular}{| c |}
		% 			\hline
		% 			\textbf{b-veto $m_{T}\geq100$ Control Region} \\ \hline \hline
		% 			1 \tauhad \\
		% 			$\pt^{\tau} > 40 \: GeV $  \\
		% 			$\geq 3$ jets \\
		% 			$\pt^{jet} > 25 \: GeV$ \\
		% 			$\Etm > 150 GeV$ \\
		% 			$m_{T}(\tau, \Etm) > 100 \: GeV$ \\
		% 			b veto \\
		% 			$\ell$ veto \\
		% 			\hline
		% 		\end{tabular}
		% 		\subcaption{Fake $j \to \tau$ enriched region}
		% 	\end{subtable}

		% 	\caption{Control region definitions for the \taujets subchannel.}
		% 	\label{tab:taujet-control-regions}
		% \end{table}

		% \begin{table}[!thp]
		% 	\begin{subtable}[c]{0.45\textwidth}
		% 		\centering
		% 		\begin{tabular}{| c |}
		% 			\hline
  %       	\textbf{Dilepton-btag CR} \\ \hline \hline
  %         $\tau$ veto \\
  %         $n\geq 1$ jets \\
  %         $\pt^{jet} > 25 GeV$ \\
  %         $\geq 1$ \bjets \\
  %         $\Etm > 50 GeV$ \\
  %         1 $e$ \\
  %         1 $\mu$ \\
  %         \hline
		% 		\end{tabular}
		% 		\subcaption{\ttbar and single top modeling}
		% 	\end{subtable}
		% 	\begin{subtable}[c]{.45\textwidth}
		% 		\centering
		% 		\begin{tabular}{| c |}
		% 			\hline
	 %        \textbf{b-veto CR} \\ \hline \hline
  %         1 \tauhad \\
  %         $\pt^{\tau} > 30 \: GeV $  \\
  %         1 $e (\mu) $ \\
  %         Veto $\mu \:(e)$ \\
  %         Opposite sign $\tau$ $e \: (\mu)$ \\
  %         $\geq 1$ jets  \\
  %         $\pt^{jet} > 25 GeV$ \\
  %         $\Etm > 50 GeV$ \\
  %         1 tight $e \: (\mu)$ \\
  %         \hline					
		% 		\end{tabular}
		% 		\subcaption{Close to signal region}
		% 	\end{subtable}
		% 	\begin{subtable}[c]{.45\textwidth}
		% 		\centering
		% 		\begin{tabular}{| c |}
		% 			\hline
  %       	\textbf{Zee CR} \\ \hline \hline
  %         1 \tauhad \\
  %         $\pt^{\tau} > 30 \: GeV $  \\
  %         veto $\mu$ \\
  %         Opposite sign $\tau$ $e$ \\
  %         $\geq 1$ jets \\
  %         $\pt^{jet} > 25 GeV$ \\
  %         \bjet veto \\
  %         $\Etm > 50 GeV$ \\
  %         1 $e$\\
  %         $40 < mass(\tau,e) < 140 GeV$ \\
  %         \hline
		% 		\end{tabular}
		% 		\subcaption{Fake $\ell \to \tau$ enriched region}
		% 	\end{subtable}
		% 	\begin{subtable}[c]{.45\textwidth}
		% 		\centering
		% 		\begin{tabular}{| c |}
		% 			\hline
	 %        \textbf{Same Sign CR} \\ \hline \hline
  %         1 \tauhad \\
  %         $\pt^{\tau} > 30 \: GeV $  \\
  %         Same sign $\tau \: e(\mu)$ \\
  %         Veto $\mu\:(e)$ \\
  %         $\geq 1 jets$ \\
  %         $\pt^{jet} > 25 GeV$ \\
  %         $\Etm > 50 GeV$  \\
  %         1 tight $e \: (\mu)$ \\
  %         \hline
		% 		\end{tabular}
		% 		\subcaption{Fake $j \to \tau$ enriched region}
		% 	\end{subtable}

		% 	\caption{Control region definitions for the \taulep subchannel.}
		% 	\label{tab:taulep-control-regions}
		% \end{table}

		\begin{table}[!thp]
			\begin{tabular}{| c | c | c | c | c |} \hline
														& \ttbar CR 		& W+Jets CR 		& b-veto CR 		& b-veto $m_{T} > 100$ CR 						 	\\ \hline
				Number of \tauhad 	& 1 						& 1 						& 0 						& 0 																		\\ \hline
				$\pt^{\tau}$ 				& $ > 40$ GeV 	& $ > 40$ GeV 	& $ > 40$ GeV 	& $ > 40$ GeV 													\\ \hline
				Number of jets 			& $\geq 3$ 			& $\geq 3$ 			& $\geq 3$ 			& $\geq 3$ 															\\ \hline
				$\pt^{jet}$ 				& $\geq 25$ GeV & $\geq 25$ GeV & $\geq 25$ GeV & $\geq 25$ GeV 												\\ \hline
				Number of \bjets		& $\geq 2$ 			& 0 						& 0 						& 0 																		\\ \hline
				Number of $\ell$ 		& 0 						& 0 						& 0 						& 0 																		\\ \hline
				\Etm 								& $> 150$ GeV 	& $> 150$ GeV 	& $> 150$ GeV 	& $> 150$ GeV 													\\ \hline
				$m_{T}(\tau, \Etm)$	& $< 100$ GeV 	& $< 100$ GeV 	& $> 50 GeV$ 		& $> 100$ GeV 													\\ \hline
				Type of modeling 		& \ttbar 				& W+Jets 				& Close to SR 	& Fake $j \to \tau$ enriched 						\\ \hline
			\end{tabular}
			\caption{Control region definitions for the \taujets subchannel.}
			\label{tab:taujet-control-regions}
		\end{table}


		\begin{table}[!thp]
			\begin{tabular}{| c | c | c | c | c |} \hline
																& Dilepton-btag CR 				& Zee CR 												& b-veto CR 					& Same Sign $(\tau,\ell)$ CR 						 							\\ \hline
				Number of \tauhad 			& 0 											& 1 														& 0 									& 0 																		\\ \hline
				$\pt^{\tau}$ 						& $ > 30$ GeV 						& $ > 30$ GeV 									& $ > 30$ GeV 				& $ > 30$ GeV 													\\ \hline
				Number of jets 					& $\geq 1$ 								& $\geq 1$ 											& $\geq 1$ 						& $\geq 1$ 															\\ \hline
				$\pt^{jet}$ 						& $\geq 25$ GeV 					& $\geq 25$ GeV 								& $\geq 25$ GeV 			& $\geq 25$ GeV 												\\ \hline
				Number of \bjets				& $\geq 1$ 								& 0 														& 0 									& $\geq 1$ 															\\ \hline
				Number of $\ell$ 				& 2 (1 $e$, 1 $\mu$)			& 1 $e$													& 1 tight $e$ ($\mu$)	& 1 tight $e$ ($\mu$)										\\ \hline
				\Etm 										& $> 50$ GeV 							& $> 50$ GeV 										& $> 50$ GeV 					& $> 50$ GeV 														\\ \hline
				$\mathrm{mass}(\tau,e)$	& N/A 										& $> 40$; $< 140$ GeV 					& N/A 								& N/A 																	\\ \hline
				Type of modeling 				& \ttbar and single-top 	& Fake $\ell \to \tau$ enriched	& Close to SR 				& Fake $j \to \tau$ enriched 						\\ \hline
			\end{tabular}
			\caption{Control region definitions for the \taulep subchannel.}
			\label{tab:taulep-control-regions}
		\end{table}



		As seen in Table \ref{tab:backgrounds} misidentified objects appearing as \tauhad candidates comprise a significant portion of the total background. Fakes arising from $\ell \to \tau$ misidentification are well modeled in \gls{MC} simulations and are reweighted with scale factors provided by the ATLAS $\tau$ combined performance group. The mass of the \tauhad electron system can be seen in Figure \ref{fig:zee-mass} as verification of fake $\ell \to \tau$ modeling. Fakes due to $j \to \tau$ misidentification have poorly misunderstood systematic uncertainties associated with the fake \tauhad object and limited statistics of simulated events. To combat this, a data driven method is used to extract a scaling constant referred to as a fake factor.

		\begin{figure}[!thp]
			\centering
			\includegraphics[width=.5\textwidth,keepaspectratio=true]{chapters/chapter6_HPlus/images/taulep/tau_0_lep_0_mass_ZEE.png}
			\caption{Mass of $\tau$ - e system in the Zee control region. All systematics except for \ttbar theory uncertainties are included.}
			\label{fig:zee-mass}
		\end{figure}

		In the \taulep final state a significant portion of $j \to \tau$ fakes come from misidentifying \tauhad candidates in W+jets events that contain a true $\ell$ in the W decay and have a misidentified jet as a \tauhad. Fakes of this manner also arise from \acrshort{QCD}-like multi-jet interactions. The \gls{FF} method used to estimate the amount of expected fake \tauhad objects that pass the \tauhad identification procedure is described in Section \ref{ssec:reco-tau}. This method applies weights, or fake factors, to a subset of "anti-\tauhad" objects that have failed the selection and identification criteria in the signal region. A control region is defined to be rich in anti-\tauhad objects, where the \tauhad candidates fail the loose $\tau$ working point but have a small, non-zero $\tau$ identification \gls{RNN} score. The \gls{FF} and number of events with misidentified \tauhad objects $(N^{\tau}_{fakes})$ are defined as:
		\begin{equation}\label{eqn:ff}\begin{split}
		FF = \frac{ N^{\tau-id} }{N^{anti-\tau-id}} \\
		N_{fakes}^{\tau} = N^{anti-\tau}_{fakes} \times FF
		\end{split}\end{equation}
		Both of these values are then corrected for \tauhad candidates matching a true hadronic $\tau$ at generator level:
		\begin{equation}\label{eqn:ff-corrected}\begin{split}
		N^{\tau-id} = N^{\tau-id}(Data) - N^{\tau-id}(MC) \\
		N^{anti-\tau}_{fakes} = N^{anti-\tau}(Data) - N^{anti-\tau}_{true} (MC)
		\end{split}\end{equation}
		Two \glspl{CR} are created, one to capture the multi-jet (MJ) fakes and the other to study the W+jets fakes. The MJ \gls{CR} uses the \taujets signal region definition with an additional b-veto and an $\Etm < 80$ GeV cut. The W+jets \gls{CR}\footnote{This W+jets \gls{CR} is not the one defined in Table \ref{tab:taujet-control-regions}. This is a new region used to extract the fake factors.} uses the \taulep signal region definition with a b-veto, no \Etm cut, and a cut on the transverse mass of the $\ell$-\Etm system of $60 < m_{T}(\ell, \Etm)<160$ GeV.
		The \gls{FF} in the signal region is defined as 
		\begin{equation}\label{eqn:ff-sig}
		FF_{sig} = \alpha_{MJ} \times FF_{MJ} + (1 - \alpha_{MJ}) \times FF_{W+jets}
		\end{equation}
		where $\alpha$ is taken from a template fit of the $\tau$-ID score distributions of the anti-$\tau$s using template shapes from the anti-$\tau$ distributions in the MJ and W+jets control regions. In the signal regions, the number of events containing fake-\tauhad candidates is defined as
		\begin{equation}\label{eqn:nfakethad}
			N_{fake-\tau} = FF_{sig} \times N_{anti-\tau} 
		\end{equation}
		Figure \ref{fig:FF_COM} shows \gls{FF} plotted in each control region for 1-prong and 3-prong \tauhad binned in $\pt{\tau}$; extracted $\alpha$ values and their fits can be seen in Appendix \ref{app:fake-factors}.
		\begin{figure}[h!]
		  \begin{center}
		    \includegraphics[width=0.45\textwidth]{chapters/chapter6_HPlus/images/FFs/FFs_COM_inclusive__taujet.png} \qquad
		    \includegraphics[width=0.45\textwidth]{chapters/chapter6_HPlus/images/FFs/FFs_COM_inclusive__taulep.png} 
		  \end{center}
		  \caption{
		Combined \gls{FF} for the \taujets b-veto $\mathrm{m_{T}}>$100 control region, \taujets signal region, $\tau$+electron(muon) with same-sign control region and the \taulep signal region. Error bars represent systematic uncertainties of the method. 
		}
		  \label{fig:FF_COM}
		\end{figure}


		To verify background modeling, the \Etm distributions in each of the control regions are plotted with final scale factors including fake factors in Figure \ref{fig:bkg-met-taujets} (\taujets) and Figures \ref{fig:bkg-met-taulep-1} - \ref{fig:bkg-met-taulep-2} (\taulep). Similar distributions of the $\tau$ \pt can be seen in Figure \ref{fig:bkg-tau-pt-taujets} (\taujets) and Figures \ref{fig:bkg-tau-pt-taulep-1} - \ref{fig:bkg-tau-pt-taulep-2}. These plots include a ratio of reconstructed data events and simulated \gls{MC} events bin by bin to ensure proper modeling across variable shapes. More background modeling plots can be seen in Appendix \ref{app:valid-plots}. Tables \ref{tab:expected_yields_taujets}, \ref{tab:expected_yields_taue} and \ref{tab:expected_yields_taumu} show the expected event yields and the effect of each selection cut on the background sources as well as selected \mHpm values of 110, 170, and 1000 GeV. At the time of writing, the analysis is still blinded so data yields are not known.

		\begin{figure}[!thp]
			\begin{center}    
			% \includegraphics[width=0.45\textwidth]{chapters/chapter6_HPlus/images/taujets/met_et_TAUJET_PRESEL.png} \\
			\includegraphics[width=0.45\textwidth]{chapters/chapter6_HPlus/images/taujets/met_et_TTBAR.png}
			\includegraphics[width=0.45\textwidth]{chapters/chapter6_HPlus/images/taujets/met_et_WJETS.png} \\
			\includegraphics[width=0.45\textwidth]{chapters/chapter6_HPlus/images/taujets/met_et_BVETO.png}
			\includegraphics[width=0.45\textwidth]{chapters/chapter6_HPlus/images/taujets/met_et_BVETO_MT100.png} \\
			\end{center}
			\caption{
			Comparison between the predicted and the measured \Etm distributions in various control regions defined for the \taujets channel. The uncertainty band includes both statistical and systematic uncertainties on the background prediction. 
			}
			\label{fig:bkg-met-taujets}
		\end{figure}

		\begin{figure}[!thp]
			\begin{center}    
			% \includegraphics[width=0.45\textwidth]{chapters/chapter6_HPlus/images/taujets/met_et_TAUJET_PRESEL.png} \\
			\includegraphics[width=0.45\textwidth]{chapters/chapter6_HPlus/images/taujets/tau_0_pt_TTBAR.png}
			\includegraphics[width=0.45\textwidth]{chapters/chapter6_HPlus/images/taujets/tau_0_pt_WJETS.png} \\
			\includegraphics[width=0.45\textwidth]{chapters/chapter6_HPlus/images/taujets/tau_0_pt_BVETO.png}
			\includegraphics[width=0.45\textwidth]{chapters/chapter6_HPlus/images/taujets/tau_0_pt_BVETO_MT100.png} \\
			\end{center}
			\caption{
			Comparison between the predicted and the measured $\tau$ \pt distributions in various control regions defined for the \taujets channel. The uncertainty band includes both statistical and systematic uncertainties on the background prediction. 
			}
			\label{fig:bkg-tau-pt-taujets}
		\end{figure}

		\begin{figure}[!thp]
			\begin{center}    
			% \includegraphics[width=0.45\textwidth]{chapters/chapter6_HPlus/images/taulep/met_et_TAULEP_PRESEL.png} \\
			% \includegraphics[width=0.45\textwidth]{chapters/chapter6_HPlus/images/taulep/met_et_DILEP_BTAG.png}
			% \includegraphics[width=0.45\textwidth]{chapters/chapter6_HPlus/images/taulep/met_et_ZEE.png} \\
			\includegraphics[width=0.45\textwidth]{chapters/chapter6_HPlus/images/taulep/met_et_TAUEL_BVETO.png} 
			\includegraphics[width=0.45\textwidth]{chapters/chapter6_HPlus/images/taulep/met_et_TAUMU_BVETO.png} \\
			\includegraphics[width=0.45\textwidth]{chapters/chapter6_HPlus/images/taulep/met_et_SS_TAUEL.png} 
			\includegraphics[width=0.45\textwidth]{chapters/chapter6_HPlus/images/taulep/met_et_SS_TAUMU.png} \\
			\end{center}
			\caption{
			Comparison between the predicted and the measured \Etm distributions in various control regions defined for the \taulep channel. The uncertainty band includes both statistical and systematic uncertainties on the background prediction. 
			}
			\label{fig:bkg-met-taulep-1}
		\end{figure}

		\begin{figure}[!thp]
			\begin{center}    
			% \includegraphics[width=0.45\textwidth]{chapters/chapter6_HPlus/images/taulep/met_et_TAULEP_PRESEL.png} \\
			\includegraphics[width=0.45\textwidth]{chapters/chapter6_HPlus/images/taulep/met_et_DILEP_BTAG.png}
			\includegraphics[width=0.45\textwidth]{chapters/chapter6_HPlus/images/taulep/met_et_ZEE.png} \\
			% \includegraphics[width=0.45\textwidth]{chapters/chapter6_HPlus/images/taulep/met_et_TAUEL_BVETO.png} 
			% \includegraphics[width=0.45\textwidth]{chapters/chapter6_HPlus/images/taulep/met_et_TAUMU_BVETO.png} \\
			% \includegraphics[width=0.45\textwidth]{chapters/chapter6_HPlus/images/taulep/met_et_SS_TAUEL.png} 
			% \includegraphics[width=0.45\textwidth]{chapters/chapter6_HPlus/images/taulep/met_et_SS_TAUMU.png} \\
			\end{center}
			\caption{
			Comparison between the predicted and the measured \Etm distributions in various control regions defined for the \taulep channel. The uncertainty band includes both statistical and systematic uncertainties on the background prediction. 
			}
			\label{fig:bkg-met-taulep-2}
		\end{figure}

		\begin{figure}[!thp]
			\begin{center}    
			% \includegraphics[width=0.45\textwidth]{chapters/chapter6_HPlus/images/taulep/met_et_TAULEP_PRESEL.png} \\
			% \includegraphics[width=0.45\textwidth]{chapters/chapter6_HPlus/images/taulep/met_et_DILEP_BTAG.png}
			% \includegraphics[width=0.45\textwidth]{chapters/chapter6_HPlus/images/taulep/met_et_ZEE.png} \\
			\includegraphics[width=0.45\textwidth]{chapters/chapter6_HPlus/images/taulep/tau_0_pt_TAUEL_BVETO.png} 
			\includegraphics[width=0.45\textwidth]{chapters/chapter6_HPlus/images/taulep/tau_0_pt_TAUMU_BVETO.png} \\
			\includegraphics[width=0.45\textwidth]{chapters/chapter6_HPlus/images/taulep/tau_0_pt_SS_TAUEL.png} 
			\includegraphics[width=0.45\textwidth]{chapters/chapter6_HPlus/images/taulep/tau_0_pt_SS_TAUMU.png} \\
			\end{center}
			\caption{
			Comparison between the predicted and the measured $\tau$ \pt distributions in various control regions defined for the \taulep channel. The uncertainty band includes both statistical and systematic uncertainties on the background prediction. 
			}
			\label{fig:bkg-tau-pt-taulep-1}
		\end{figure}

		\begin{figure}[!thp]
			\begin{center}    
			% \includegraphics[width=0.45\textwidth]{chapters/chapter6_HPlus/images/taulep/met_et_TAULEP_PRESEL.png} \\
			\includegraphics[width=0.45\textwidth]{chapters/chapter6_HPlus/images/taulep/tau_0_pt_DILEP_BTAG.png}
			\includegraphics[width=0.45\textwidth]{chapters/chapter6_HPlus/images/taulep/tau_0_pt_ZEE.png} \\
			% \includegraphics[width=0.45\textwidth]{chapters/chapter6_HPlus/images/taulep/met_et_TAUEL_BVETO.png} 
			% \includegraphics[width=0.45\textwidth]{chapters/chapter6_HPlus/images/taulep/met_et_TAUMU_BVETO.png} \\
			% \includegraphics[width=0.45\textwidth]{chapters/chapter6_HPlus/images/taulep/met_et_SS_TAUEL.png} 
			% \includegraphics[width=0.45\textwidth]{chapters/chapter6_HPlus/images/taulep/met_et_SS_TAUMU.png} \\
			\end{center}
			\caption{
			Comparison between the predicted and the measured $\tau$ \pt distributions in various control regions defined for the \taulep channel. The uncertainty band includes both statistical and systematic uncertainties on the background prediction. 
			}
			\label{fig:bkg-tau-pt-taulep-2}
		\end{figure}

		\begin{table}[!thp]
			 \begin{center}
			 \small
			\resizebox{\textwidth}{!}{
			\begin{tabular}{|l|r@{ $\pm$ }l|r@{ $\pm$ }l|r@{ $\pm$ }l|r@{ $\pm$ }l|}
			\hline
			Selection                                   & \multicolumn{2}{c|}{$t\bar{t}$} & \multicolumn{2}{c|}{Single-top-quark}               & \multicolumn{2}{c|}{$W \to \tau\nu$} & \multicolumn{2}{c|}{$Z \to \tau\tau$}  \\ 
			\hline
			Trigger and skim                            & 342315.32 & 205.10 & 38154.55 & 66.40 & 296087.10 & 539.63 & 52976.26 & 148.89 \\
			Loose tau, $p_\mathrm{T}^{\tau}>40$ GeV   & 231610.69 & 169.16 (68\%) & 26454.25 & 55.97 (69\%) & 197122.72 & 436.12 (67\%) & 37617.08 & 119.38 (71\%) \\
			Medium tau             & 133237.32 & 127.99 (58\%) & 16879.04 & 44.91 (64\%) & 142441.76 & 319.72 (72\%) & 27017.86 & 101.54 (72\%) \\
			Lepton veto  & 114888.70 & 118.99 (86\%) & 15450.09 & 42.79 (92\%) & 142390.47 & 319.70 (100\%) & 23911.34 & 97.78 (89\%) \\
			$\ge 1$ b-jet                    & 93242.48 & 106.97 (81\%) & 11711.18 & 37.03 (76\%) & 11788.34 & 62.30 (8\%) & 2923.99 & 23.68 (12\%) \\
			$E_\mathrm{T}^\mathrm{miss}>150$ GeV\       & 48457.24 & 78.37 (52\%) & 7129.31 & 29.82 (61\%) & 7160.34 & 43.20 (61\%) & 1194.29 & 11.05 (41\%) \\
			$m_\mathrm{T}>50 GeV$ (SR)                          & 18369.33 & 48.16 (38\%) & 2276.08 & 16.69 (32\%) & 1972.76 & 23.54 (28\%) & 241.05 & 5.47 (20\%) \\
			\hline
			Selection                                   &   \multicolumn{2}{c|}{Diboson ($WW, WZ, ZZ$)}     & \multicolumn{2}{c|}{Misidentified $e,\,\mu \to \tauhad$}  & \multicolumn{2}{c|}{Misidentified $\mbox{jet} \to \tauhad$} & \multicolumn{2}{c|}{All backgrounds}  \\ 
			\hline
			Trigger and skim                            & 13094.19 & 53.23 & 708734.11 & 890.75 & 3195982.69 & 560.82 & 4647344.22 & 1212.70 \\
			Loose tau, $p_\mathrm{T}^{\tau}>40$ GeV   & 9387.33 & 45.05 (72\%) & 375566.87 & 328.44 (53\%) & 1645126.81 & 354.50 (51\%) & 2522885.75 & 686.86 (54\%) \\
			Medium tau             & 6591.35 & 37.96 (70\%) & 2924.49 & 84.42 (1\%) & 198869.55 & 137.33 (12\%) & 527961.36 & 397.94 (21\%) \\
			Lepton veto  & 5976.46 & 37.70 (91\%) & 2247.49 & 83.45 (77\%) & 191880.62 & 134.54 (96\%) & 496745.16 & 392.74 (94\%) \\
			$\ge 1$ b-jet                    & 625.84 & 10.33 (10\%) & 1194.29 & 40.93 (53\%) & 38868.36 & 61.73 (20\%) & 160354.47 & 151.16 (32\%) \\
			$E_\mathrm{T}^\mathrm{miss}>150$ GeV\       & 414.89 & 8.66 (66\%) & 428.96 & 7.80 (36\%) & 3009.41 & 21.14 (8\%) & 67794.44 & 97.99 (42\%) \\
			$m_\mathrm{T}>50 GeV$ (SR)                          & 133.30 & 4.67 (32\%) & 327.51 & 6.82 (76\%) & 2490.58 & 17.35 (83\%) & 25810.61 &59.60 (38\%) \\
			\hline\
			Selection                                        & \multicolumn{2}{c|}{$\Hpm$ $\phantom{0}$(110~GeV)} & \multicolumn{2}{c|}{$\Hpm$ $\phantom{0}$(170~GeV)} & \multicolumn{2}{c|}{$\Hpm$ $\phantom{0}$(1000~GeV)} & \multicolumn{2}{c|}{Data (\LUMI )}  \\ 
			\hline
			Trigger and skim                            & 1951.47 & 11.05 & 4992.07 & 19.02 & 34812.84 & 94.47 & XXX & XXX \\
			Loose tau, $p_\mathrm{T}^{\tau}>40$ GeV   & 1554.18 & 9.87 (80\%) & 4047.73 & 17.14 (81\%) & 27900.09 & 84.24 (80\%) & XXX & XXX\\
			Medium tau             & 1016.12 & 7.99 (65\%) & 2890.75 & 14.52 (71\%) & 19994.71 & 72.07 (72\%) & XXX  & XXX\\
			Lepton veto  & 901.33 & 7.53 (89\%) & 2694.19 & 14.02 (93\%) & 18343.94 & 68.93 (92\%) & XXX & XXX\\
			$\ge 1$ b-jet                    & 729.40 & 6.76 (81\%) & 2075.45 & 12.29 (77\%) & 13611.31 & 60.35 (74\%) & XXX & XXX \\
			$E_\mathrm{T}^\mathrm{miss}>150$ GeV\       & 425.52 & 5.25 (58\%) & 1333.74 & 10.03 (64\%) & 12797.45 & 58.63 (94\%) & XXX & XXX \\
			$m_\mathrm{T}>50 GeV$ (SR)                          & 264.68 & 4.13 (62\%) & 1044.07 & 8.86 (78\%) & 12728.36 & 58.47 (99\%) & XXX & XXX \\
			\hline
			\end{tabular}}
			 \caption{\label{tab:expected_yields_taujets}
			     Expected event yields and efficiencies after cumulative selection cuts and comparison with \LUMI of 
			     data for \taujets sub-channel.
			     The values shown for the signal correspond to $\sigma(\pp \to [b]t\Hpm) \times Br(\HpmLong)=1$~pb.
			     Statistical uncertainties are shown.}
			%     \textcolor{red}{
			%     Both the statistical and systematic uncertainties (Section~\ref{sec:systematic_uncertainties}) are shown for the final selection cut in 
			%the last row.
			%     } % end textcolor{red}
			 \end{center}
	   \end{table}

		\begin{table}[!thp]
			\begin{center}
			\small
			\resizebox{\textwidth}{!}{
			\begin{tabular}{|l|r@{ $\pm$ }l|r@{ $\pm$ }l|r@{ $\pm$ }l|r@{ $\pm$ }l|}
			\hline
			Selection                                   & \multicolumn{2}{c|}{$t\bar{t}$} & \multicolumn{2}{c|}{Single-top-quark}               & \multicolumn{2}{c|}{$W \to \tau\nu$} & \multicolumn{2}{c|}{$Z \to \tau\tau$}  \\ 
			\hline
			Trigger and skim                            & 410886.20 & 232.60 & 40098.96 & 68.91 & 1504.99 & 67.50 & 214323.27 & 1518.06 \\
			Loose tau, $p_\mathrm{T}^{\tau}>30$ GeV   & 355267.50 & 216.28 (86\%) & 34634.86 & 64.10 (86\%) & 1280.89 & 61.15 (85\%) & 184446.15 & 1411.31 (86\%) \\
			Medium tau             & 179348.02 & 154.01 (50\%) & 17098.32 & 46.93 (49\%) & 426.20 & 44.28 (33\%) & 128420.09 & 1199.81 (70\%) \\
			Tight electron, $p_\mathrm{T}^{e}>30$ GeV  & 80461.17 & 104.31 (45\%) & 6920.81 & 30.31 (40\%) & 112.68 & 16.71 (26\%) & 39581.52 & 621.39 (31\%) \\
			Electron and tau with OS                    & 79604.45 & 103.76 (99\%) & 6837.95 & 30.14 (99\%) & 83.19 & 15.00 (74\%) & 39164.25 & 618.78 (99\%) \\
			$E_\mathrm{T}^\mathrm{miss}>50$ GeV\       & 53813.81 & 85.27 (68\%) & 4547.93 & 24.61 (67\%) & 38.97 & 6.28 (47\%) & 12448.80 & 218.27 (32\%) \\
			$\ge$1 b-jet (SR)                          & 43814.66 & 76.84 (81\%) & 3260.70 & 20.81 (72\%) & 2.41 & 0.56 (6\%) & 913.61 & 20.42 (7\%) \\
			\hline
			Selection                                   &   \multicolumn{2}{c|}{Diboson ($WW, WZ, ZZ$)}     & \multicolumn{2}{c|}{Misidentified $e,\,\mu \to \tauhad$}  & \multicolumn{2}{c|}{Misidentified $\mbox{jet} \to \tauhad$} & \multicolumn{2}{c|}{All backgrounds}  \\ 
			\hline
			Trigger and skim                            & 18533.34 & 23.85 & 3537419.24 & 4942.14 & 2403411.34 & 714.76 & 6626177.34 & 5225.34 \\
			Loose tau, $p_\mathrm{T}^{\tau}>30$ GeV   & 16060.23 & 22.20 (87\%) & 1939418.21 & 2933.87 (55\%) & 2027356.60 & 588.57 (84\%) & 4558464.44 & 3316.76 (69\%) \\
			Medium tau                    & 10957.22 & 17.56 (68\%) & 56819.79 & 851.44 (3\%) & 443714.51 & 304.58 (22\%) & 836784.15 & 1511.77 (18\%) \\
			Tight electron, $p_\mathrm{T}^{e}>30$ GeV           & 3619.35 & 10.40 (33\%) & 20011.76 & 495.63 (35\%) & 152485.84 & 182.97 (34\%) & 303193.13 & 823.07 (36\%) \\
			Electron and tau with OS                         & 3122.54 & 9.91 (86\%) & 17202.16 & 452.06 (86\%) & 95723.92 & 147.37 (63\%) & 241738.46 & 788.01 (80\%) \\
			$E_\mathrm{T}^\mathrm{miss}>50$ GeV\       & 1903.40 & 7.70 (61\%) & 3616.05 & 157.27 (21\%) & 31529.78 & 79.00 (33\%) & 107898.73 & 294.27 (45\%) \\
			$\ge$1 b-jet (SR)                          & 73.21 & 1.53 (4\%) &  1096.64 & 24.36 (30\%) & 8773.81 & 37.64 (28\%) & 57935.04 & 93.63 (54\%) \\
			\hline\
			Selection                                        & \multicolumn{2}{c|}{$\Hpm$ $\phantom{0}$(110~GeV)} & \multicolumn{2}{c|}{$\Hpm$ $\phantom{0}$(170~GeV)} & \multicolumn{2}{c|}{$\Hpm$ $\phantom{0}$(1000~GeV)} & \multicolumn{2}{c|}{Data (\LUMI )}  \\ 
			\hline
			Trigger and skim                            & 3202.25 & 13.97 & 4116.46 & 17.10 & 5268.21 & 30.05 & XXX & XXX \\
			Loose tau, $p_\mathrm{T}^{\tau}>30$ GeV & 2793.23 & 13.05 (87\%) & 3600.92 & 15.99 (87\%) & 4313.45 & 27.22 (82\%) & XXX & XXX \\
			Medium tau   & 1915.08 & 10.81 (69\%) & 2592.01 & 13.58 (72\%) & 3132.37 & 23.23 (73\%) & XXX & XXX \\
			Tight electron, $p_\mathrm{T}^{e}>30$ GeV\  & 852.54 & 7.40 (45\%) & 1107.30 & 9.12 (43\%) & 1356.64 & 15.63 (43\%) & XXX & XXX \\
			Electron and tau with OS                    & 845.17 & 7.37 (99\%) & 1094.53 & 9.06 (99\%) & 1330.08 & 15.48 (98\%) & XXX & XXX  \\
			$E_\mathrm{T}^\mathrm{miss}>50$ GeV\          & 547.00 & 5.94 (65\%) & 837.08 & 7.93 (76\%) & 1302.14 & 15.29 (98\%) & XXX & XXX \\
			$\ge$1 b-jet (SR)                            & 440.11 & 5.33 (80\%) & 613.35 & 6.78 (73\%) & 956.26 & 13.50 (73\%) & XXX & XXX \\
			\hline
			\end{tabular}}
			\caption{\label{tab:expected_yields_taue}
			  Expected event yields and efficiencies after cumulative selection cuts and comparison with \LUMI of
			 data for \tauel channel. 
			 The values shown for the signal correspond to $\sigma(\pp \to [b]t\Hpm) \times Br(\HpmLong)=1$~pb.
			 Statistical uncertainties are shown.}
			 % The values shown for the signal correspond to the cross sections predicted at $\tan\beta = 40$ in the hMSSM benchmark scenario.
			%     \textcolor{red}{
			%     Both the statistical and systematic uncertainties (Section~\ref{sec:systematic_uncertainties}) are shown for the final selection cut in 
			%the last row.
			%    } % end textcolor{red}
			%\textcolor{red}{This table needs to be updated!}
			\end{center}
		\end{table}

		\begin{table}[!thp]
			 \begin{center}
			 \small
			\resizebox{\textwidth}{!}{
			\begin{tabular}{|l|r@{ $\pm$ }l|r@{ $\pm$ }l|r@{ $\pm$ }l|r@{ $\pm$ }l|}
			\hline
			Selection                                   & \multicolumn{2}{c|}{$t\bar{t}$} & \multicolumn{2}{c|}{Single-top-quark}               & \multicolumn{2}{c|}{$W \to \tau\nu$} & \multicolumn{2}{c|}{$Z \to \tau\tau$}  \\ 
			\hline
			Trigger and skim                            & 410886.20 & 232.60 & 40098.96 & 68.91 & 1504.99 & 67.50 & 214323.26 & 1518.06 \\
			Loose tau, $p_\mathrm{T}^{\tau}>30$ GeV   & 355267.50 & 216.28 (86\%) & 34634.86 & 64.10 (86\%) & 1280.89 & 61.15 (85\%) & 184446.15 & 1411.31 (86\%) \\
			Medium tau             & 179348.02 & 154.01 (50\%) & 17098.32 & 46.93 (49\%) & 426.20 & 44.28 (33\%) & 128420.09 & 1199.81 (70\%) \\
			Tight muon, $p_\mathrm{T}^{e}>30$ GeV  & 83293.59 & 103.84 (46\%) & 8704.78 & 33.09 (51\%) & 44.56 & 12.50 (10\%) & 57089.24 & 811.41 (44\%) \\
			Muon and tau with OS                    & 82852.63 & 103.57 (99\%) & 8653.92 & 33.01 (99\%) & 42.72 & 12.17 (96\%) & 56860.99 & 810.34 (100\%) \\
			$E_\mathrm{T}^\mathrm{miss}>50$ GeV\       & 55250.05 & 84.70 (67\%) & 5516.70 & 26.38 (64\%) & 7.66 & 2.72 (18\%) & 13803.33 & 220.41 (24\%) \\
			$\ge$1 b-jet (SR)                          & 44490.69 & 75.96 (81\%) & 3874.57 & 22.06 (70\%) & 0.07 & 0.12 (1\%) & 845.89 & 22.07 (6\%) \\
			\hline
			Selection                                   &   \multicolumn{2}{c|}{Diboson ($WW, WZ, ZZ$)}     & \multicolumn{2}{c|}{Misidentified $e,\,\mu \to \tauhad$}  & \multicolumn{2}{c|}{Misidentified $\mbox{jet} \to \tauhad$} & \multicolumn{2}{c|}{All backgrounds}  \\ 
			\hline
			Trigger and skim                            & 18533.34 & 23.85 & 3537419.44 & 4942.14 & 2403411.34 & 714.76 & 6626177.54 & 5225.34 \\
			Loose tau, $p_\mathrm{T}^{\tau}>30$ GeV   & 16060.23 & 22.20 (87\%) & 1939418.20 & 2933.87 (55\%) & 2027356.60 & 588.57 (84\%) & 4558464.43 & 3316.76 (69\%) \\
			Medium tau                    & 10957.22 & 17.56 (68\%) & 56819.79 & 851.44 (3\%) & 443714.51 & 304.58 (22\%) & 836784.15 & 1511.77 (18\%) \\
			Tight muon, $p_\mathrm{T}^{e}>30$ GeV           & 5709.54 & 12.47 (52\%) & 30268.99 & 623.62 (53\%) & 199706.03 & 205.19 (45\%) & 384816.73 & 1049.56 (46\%) \\
			Muon and tau with OS                         & 4960.17 & 11.89 (87\%) & 27996.60 & 611.93 (92\%) & 131414.33 & 171.52 (66\%) & 312781.35 & 1035.68 (81\%) \\
			$E_\mathrm{T}^\mathrm{miss}>50$ GeV\       & 2799.13 & 8.89 (56\%) & 3766.50 & 161.20 (13\%) & 42513.90 & 90.25 (32\%) & 123657.27 & 301.10 (40\%) \\
			$\ge$1 b-jet (SR)                          & 81.32 & 1.53 (3\%) & 1074.28 & 15.90 (29\%) & 8558.21 & 37.23 (20\%) & 58925.03 & 91.57 (48\%) \\
			\hline\
			Selection                                        & \multicolumn{2}{c|}{$\Hpm$ $\phantom{c|}$(110~GeV)} & \multicolumn{2}{c|}{$\Hpm$ $\phantom{0}$(170~GeV)} & \multicolumn{2}{c|}{$\Hpm$ $\phantom{0}$(1000~GeV)} & \multicolumn{2}{c|}{Data (\LUMI )}  \\ 
			\hline
			Trigger and skim                            & 3202.25 & 13.97 & 4116.46 & 17.10 & 5268.21 & 30.05 & XXX & XXX \\
			Loose tau, $p_\mathrm{T}^{\tau}>30$ GeV & 2793.23 & 13.05 (87\%) & 3600.92 & 15.99 (87\%) & 4313.45 & 27.22 (82\%) & XXX & XXX \\
			Medium tau   & 1915.08 & 10.81 (69\%) & 2592.01 & 13.58 (72\%) & 3132.37 & 23.23 (73\%) & XXX & XXX \\
			Tight muon, $p_\mathrm{T}^{e}>30$ GeV\  & 889.27 & 7.17 (46\%) & 1264.26 & 9.25 (49\%) & 1499.58 & 15.93 (48\%) & XXX & XXX \\
			Muon and tau with OS                    & 884.48 & 7.15 (99\%) & 1258.52 & 9.22 (100\%) & 1485.25 & 15.88 (99\%) & XXX & XXX \\
			$E_\mathrm{T}^\mathrm{miss}>50$ GeV\          & 563.90 & 5.71 (64\%) & 946.69 & 8.01 (75\%) & 1447.52 & 15.62 (97\%) & XXX & XXX \\
			$\ge$1 b-jet (SR)                            & 447.73 & 5.07 (79\%) & 693.80 & 6.85 (73\%) & 1004.35 & 12.95 (69\%) & XXX & XXX \\
			\hline
			\end{tabular}}
			 \caption{\label{tab:expected_yields_taumu}
			     Expected event yields and efficiencies after cumulative selection cuts and comparison with \LUMI of
			     data for \taumu channel. 
			     The values shown for the signal correspond to $\sigma(\pp \to [b]t\Hpm) \times Br(\HpmLong)=1$~pb.
			     % The values shown for the signal correspond to the cross sections predicted at $\tan\beta = 40$ in the hMSSM benchmark scenario.
			     Statistical uncertainties are shown.}
			     %Both the statistical and systematic uncertainties (Section~\ref{sec:systematic_uncertainties}) are shown for the final selection cut in the last row.
			    %\textcolor{red}{This table needs to be updated!}
			 \end{center}
	   \end{table}


	\clearpage
	\section{Multivariate Analysis Techniques}\label{sec:mva}
		Once variables distributions are properly scaled and data/\gls{MC} agreement is verified, multivariate analysis techniques are employed to separate signal-like events from background-like events in the signal regions. In the previous publication (described in Section \ref{ssec:Prev Hpm}), \glspl{BDT} binned in \mHpm were used as the classifier, whereas this publication uses one \gls{PNN} for the entire \mHpm spectrum. \glspl{BDT} excel at separating linear correlations, whereas neural networks take advantage of nonlinear correlations. In the case of a \gls{PNN} the parameterized variable, here \mHpm, is taken as an input to the network in addition with other input variables. \gls{PNN}s offer the advantage of having one classifier model that can evaluate at any \mHpm value by learning how the signal event topology changes as \mHpm varies \cite{PNN}. For illustrative purposes, expected limits on $\sigma(\pp\to tb\Hpm)\times \mathrm{\cal{B}}(\Hpm \to \tau \nu)$ in both subchannels is shown comparing an optimized \gls{BDT} and an unoptimized \gls{PNN} in Figure \ref{fig:bdt-vs-pnn-expected-limits}. It is seen that the \gls{PNN} performs similarly to the \glspl{BDT} used in the previous analysis. A \gls{PNN} was chosen as the discriminator. 

		\begin{figure}
		\subfloat[\label{fig:bdt-vs-pnn-expected-limits-a}]{\includegraphics[width=.5\textwidth]{chapters/chapter6_HPlus/images/Limits/exp_limit_log_taujet.pdf}}
		\subfloat[\label{fig:bdt-vs-pnn-expected-limits-b}]{\includegraphics[width=.5\textwidth]{chapters/chapter6_HPlus/images/Limits/Exp_Limit_log_taulep.eps}}
		\caption{Comparison of performance of an optimized \gls{BDT} and an unoptimized \gls{PNN} on expected limits on $\sigma(\pp\to tb\Hpm)\times \mathrm{\cal{B}}(\Hpm \to \tau \nu)$ in the \taujets (a) and \taulep (b) signal regions. }
		\label{fig:bdt-vs-pnn-expected-limits}
		\end{figure}

		A \gls{NN} is a computing system loosely inspired by the human brain. NNs combine adaptive nonlinear basis functions in an attempt to perform a task; classification in the context of this dissertation. A NN contains layers of nodes connected to each other with an associated weight and threshold. As long as a node has output greater than the given threshold value, data will flow through that node\footnote{This is true of basic \glspl{NN}. In some cases, this dissertation included, nodes are allowed small non-zero weights (negative or positive) to retain a so called ``leaky'' node.}. Otherwise, that node is not activated and data are not sent to the next layer. The NN as a whole relies on a process called training where the node weights are varied, an accuracy is calculated based on a given loss function, the weights are then varied again and the process repeats. This is done until a preferred accuracy is reached; the final node weights are saved and new data can be evaluated. A diagram of a \gls{PNN} can be seen in Figure \ref{fig:PNN-diagram}, where the parameterized input is labeled as $\theta$. The learned function of a NN can be written as:
		\begin{equation}
		y(x) = w_{0}^{2} + \sum^{M}_{m=1}[ w^{2}_{m} \cdot h (w_{0m}^{1} + \sum^{D}_{k=1} w^{1}_{km} x_{k}  )]
		\end{equation}
		where $w$ is the neuron weights, $M$ is the number of basis functions being combined, $D$ is the number of inputs and $h$ is the activation function.

		\begin{figure}	
			\begin{center}
				\includegraphics[width=.6\textwidth,keepaspectratio=true]{chapters/chapter6_HPlus/images/PNN_Diagram.png}
			\end{center}
			\caption{\textit{Left}, individual networks with input variables $(x_{1},x_{2})$, each trained with examples with a single value of some parameter $\theta = \theta_{a}, \theta_{b}$. The individual networks are purely functions of the input variables. Performance for intermediate values of $\theta$ is not optimal nor does it necessarily vary smoothly between the networks. \textit{Right}, a single network trained with input variables $(x_{1},x_{2})$ as well as input parameter $\theta$; such a network is trained with examples at several values of the parameter $\theta$ \cite{PNN}.}
			\label{fig:PNN-diagram}
		\end{figure}	

			This analysis uses four \gls{PNN}s; the two subchannels each have separate networks for 1-prong $\tau$ and 3-prong $\tau$ events.

		\subsection{Training}\label{ssec:training}
			The training of the \gls{PNN}s used in this dissertation are done with the Keras \cite{keras} library using the TensorFlow \cite{tensorflow2015-whitepaper} library as backend. In order to increase the significance of training statistics and protect from overtraining, the $k$-fold method is used. Overtraining occurs when a NN has been fine tuned to have a high accuracy with a specific dataset and does not generalize to other datasets. To protect against this, dropout is used \cite{dropout}. The $k$-fold method divides input training samples into $k$ equally populated subsets. The $k$-th subset is trained on the other $k-1$ subsets and evaluated on the $k$-th subset. Figure \ref{fig:k-fold-diagram} shows a pictorial representation of the $k$-fold method. The standard choice of $k=5$ is used in this analysis; this ensures statistical independence of training and evaluation of the \glspl{PNN} and retaining a reasonable computational demand.

			\begin{figure}	
				\begin{center}
					\includegraphics[width=.75\textwidth,keepaspectratio=true]{chapters/chapter6_HPlus/images/kFoldDiagram_noValid.pdf}
				\end{center}
				\caption{The k-fold method for $k=5$ \cite{Burghgrave:2018uwq}.}
				\label{fig:k-fold-diagram}
			\end{figure}	

			A single \gls{PNN} training is performed on all \mHpm values at once, with the \mHpm value being taken as an input variable. For signal events, the \mHpm value from the \gls{MC} generator is given; background events are replicated 32 times (the number of simulated \mHpm points is 32) and each \mHpm value is given for each set. To avoid biasing the training due to varying statistics at each \mHpm value, the background events are weighted by a factor of $w = N^{i}_{S}/N^{i}_{B}$ where $i$ corresponds to a given \mHpm value and $N^{i}_{S}$ and $N^{i}_{B}$ are the number of signal and background events, respectively. When the \gls{PNN} is evaluated, the \mHpm value is assumed and the output is used as the discriminant at that \mHpm.

		\subsection{Input Variables Selection}\label{ssec:input-variables}
			The choice of input variables to the \glspl{PNN} is critical to the performance of the analysis. Several sets of variables were compared using expected limits as the figure of merit. All studies were performed in the \taulep signal region, as this region proves the most difficult challenge to separate signal-like events from background-like events. One such study investigated the discriminating power of two sets of input variables. Input variables set A, consisting of the four vector components of the main physics objects in each event, were compared against another set of input variables B. Tables of the two sets of input variables are shown in Table \ref{tab:taulep-input-variables-high-v-low}. These two sets of variables were chosen to investigate the difference in signal-background separation between a \gls{PNN} trained with engineered variables (Set B) and lower level kinematic information (Set A). Plots of all variables used in this study can be seen in Appendix \ref{app:valid-plots}.

			The variable $\mHpm^{Truth}$ corresponds to the \mHpm value the training and evaluation is performed at. In both cases, the variable $\Upsilon$ is used. $\Upsilon$ is a measure of the \tauhad polarization, computed by taking the asymmetry of energies carried by the charged and neutron pions from the 1-prong $\tau$ decay measured in the laboratory frame. $\Upsilon$ is defined as
			\begin{equation}\label{eqn:upsilon}
			\Upsilon = \frac{ E^{\pi^{\pm}}_{T} - E^{\pi^{0}}_{T}}{E^{\tau}_{T}} \approx 2 \frac{\pt^{\tau-\mathrm{track}}}{\pt^{\tau}} -1
			\end{equation}
			where $\pt^{\tau-\mathrm{track}}$ is the transverse momentum of the track associated with the 1-prong \tauhad candidate. As such, $\Upsilon$ is only defined for 1-prong \tauhad candidates. As demonstrated in the previous analysis, $\Upsilon$ provides a large contribution to signal-backgrounds separation at charged Higgs masses below 400 GeV \cite{hpm-previous}. This is due to $W^{-}$ bosons coupling exclusively to left-handed $\tau^{-}$ leptons in $W \to \tau \nu$ decays and $W^{+}$ bosons coupling exclusively to $\tau^{+}$ leptons. In such a case, $\Upsilon$ is expected to have a value of $-1$. Whereas in the \gls{MSSM}, a charged scalar Higgs boson would lead to an $\Upsilon$ value of $+1$ \cite{tau-polarization}. A detailed explanation and dedicated measurement of $\tau$ polarization can be seen in Reference \cite{tau-polarization}.
			% \begin{table}[!ht]
			% 	\begin{center}
			% 	\caption{List of high level kinematic variables used as input to the \gls{PNN} in the \taulep subchannel. $\Delta \phi_{X,\,\text{miss}}$ denotes the difference in azimuthal angle between a reconstructed object $X$ ($X = \tauhad,\,\bjet,\,\ell$) and the direction of the missing transverse momentum.}
			% 	\begin{tabular}{| l |}
			% 	\hline
			% 	\textbf{High Level Input Variables} \\
			% 	\hline \hline
			% 	$\Etm$  \\
			% 	$\pt^{\tau}$  \\
			% 	$\pt^{\bjet}$  \\
			% 	$\pt^{\ell}$  \\
			% 	$\Delta \phi_{\tauhad,\,\text{miss}}$  \\
			% 	$\Delta \phi_{\bjet,\,\text{miss}}$  \\
			% 	$\Delta \phi_{\ell,\,\text{miss}}$  \\
			% 	$\Delta R_{\tauhad,\,\ell}$ \\
			% 	$\Delta R_{\bjet,\,\ell}$ \\
			% 	$\Delta R_{\bjet,\,\tauhad}$ \\
			% 	$\Delta \phi_{\tauhad, \text{miss}} / \Delta \phi_{\text{jet}, \text{miss}}$  \\
			% 	$\Upsilon$ \\
			% 	$\mHpm^{Truth}$ \\ \hline
			% 	\end{tabular}
			% 	\label{tab:pnn-high-level-input-variables}
			% 	\end{center}
			% \end{table}

		 %  \begin{table}[!ht]
		 %  	\begin{center}
		 %  	\caption{ List of low level kinematic variables used as input to the \gls{PNN} in the \taulep subchannel.
		 %  	}
		  %     \begin{tabular}{| c | c | c | c |}
		  %       \multicolumn{4}{c}{\textbf{Low Level Input Variables}} \\ \hline \hline
		  %       $\pt^{\tau}$ & $\eta^{\tau}$ & $\phi^{\tau}$ & $E^{\tau}$ \\ \hline
		  %       $\pt^{\ell}$ & $\eta^{\ell}$ & $\phi^{\ell}$ & $E^{\ell}$ \\ \hline
		  %       $\pt^{\bjet}$ & $\eta^{\bjet}$ & $\phi^{\bjet}$ & $E^{\bjet}$ \\ \hline
		  %       $\pt^{jet}$ & $\eta^{jet}$ & $\phi^{jet}$ & $E^{jet}$ \\ \hline
		  %       \Etm & $\phi^{\Etm}$ & $\pt^{j_{1}}$ & $\Upsilon$  \\ \hline
		  %       $\mHpm^{Truth}$ & & & \\ \hline 
		  %       \hline
		  %       \end{tabular}
		  %       \label{tab:pnn-low-level-input-variables}
		  %       \end{center}
	   	%    \end{table}




      \begin{table}[!thp]
				\begin{subtable}[c]{0.15\textwidth}
					\centering
					\begin{tabular}{| c | c | c |}
		        \multicolumn{3}{c}{\textbf{Set A Input Variables}} \\ \hline \hline
		        $\pt^{\tau}$ & $\eta^{\tau}$ & $\phi^{\tau}$  \\ \hline
		        $\pt^{\ell}$ & $\eta^{\ell}$ & $\phi^{\ell}$  \\ \hline
		        $\pt^{\bjet}$ & $\eta^{\bjet}$ & $\phi^{\bjet}$  \\ \hline
		        $\pt^{jet}$ & $\eta^{jet}$ & $\phi^{jet}$  \\ \hline
		        \Etm & $\phi^{\Etm}$ & $\pt^{j_{1}}$  \\ \hline
		        $\Upsilon$ & $\mHpm^{Truth}$ &  \\ \hline 
	        \end{tabular}
	        \subcaption{Set A of input variables}
	      \end{subtable}

				\begin{subtable}[c]{0.25\textwidth}
					\centering
					\begin{tabular}{| l |}
						\hline
						\textbf{Set B Input Variables} \\
						\hline \hline
						$\Etm$  \\
						$\pt^{\tau}$  \\
						$\pt^{\bjet}$  \\
						$\pt^{\ell}$  \\
						$\Delta \phi_{\tau,\,\text{miss}}$  \\
						$\Delta \phi_{\bjet,\,\text{miss}}$  \\
						$\Delta \phi_{\ell,\,\text{miss}}$  \\
						$\Delta R_{\tau,\,\ell}$ \\
						$\Delta R_{\bjet,\,\ell}$ \\
						$\Delta R_{\bjet,\,\tau}$ \\
						$\Delta \phi_{\tau, \text{miss}} / \Delta \phi_{\text{jet}, \text{miss}}$  \\
						$\Upsilon$ \\
						$\mHpm^{Truth}$ \\ \hline
					\end{tabular}
					\subcaption{Set B of input variables}
				\end{subtable}
				\caption{Two sets of kinematic variables used as input to the \gls{PNN} in the \taulep subchannel. $\Delta \phi_{X,\,\text{miss}}$ denotes the difference in azimuthal angle between a reconstructed object $X$ ($X = \tau,\,\bjet,\,\ell$) and the direction of the missing transverse momentum. Distributions of all variables can be seen in Appendix \ref{app:valid-plots}.}
				\label{tab:taulep-input-variables-high-v-low}
			\end{table}

      An estimate of the impact of two sets of input variables on the expected limits on $\sigma(\pp\to tb\Hpm)\times \mathrm{\cal{B}}(\Hpm \to \tau \nu)$ is shown in \ref{fig:variable-comparison-limits}. Input variables set A was chosen as performance was similar at low \mHpm and greater at high \mHpm. An optimization of the number of layers in the \gls{PNN} and several other parameters of the \gls{PNN} is discussed in detail in Section \ref{ssec:hpo}.
			\begin{figure}	
				\begin{center}
					\includegraphics[width=.75\textwidth,keepaspectratio=true]{chapters/chapter6_HPlus/images/taulep_limits_PNN_low_lv_vs_high_lv.eps}
				\end{center}
				\caption{Expected limits comparing input variables set A and B with various depths in the \gls{PNN} architecture. X layers refers to the number of layers in the \gls{PNN}. }
				\label{fig:variable-comparison-limits}
			\end{figure}	

		\subsection{Hyperparameter Optimization}\label{ssec:hpo}

				In order to optimize the \gls{PNN}, a scan of hyperparameters and network architecture was done, referred to as \gls{HPO}. A calculated \acrfull{AUC} was used as the figure of merit. As in the normal training scheme, the $k$-fold method with $k=5$ was used to keep background modelling and classifier training statistically independent. To prevent overtraining, the early stopping method was used and the best weights kept to calculate the \gls{AUC}. The early stopping method has two parameters, $\Delta_{min}$ and patience. $\Delta_{min}$ is the minimum allowed difference in \gls{AUC} between training epochs. Once the $\Delta_{min}$ value is lower than the user defined threshold several more epochs are trained to ensure a global minima is found. The patience is this number of extra training epochs. For this dissertation $\Delta_{min}=0.00001$ and a patience of 10 were used. Due to the low signal acceptance and the increased difficulty of separating signal from background at lower \mHpm values, the \gls{PNN} was optimized for low \mHpm values.  To optimize for \gls{PNN} performance in low \Hpm mass points, a separate average taking into account only \Hpm mass values between $80$ and $500$ GeV was used as the final figure of merit. In an effort to keep the computational needs low, several small grids of hyperparameters and architecture structures were scanned. Tables \ref{Tab:taulepHPOLoss} - \ref{Tab:taulepHPOWidthDepth} show the hyperparameter grids that were searched. Here, width refers to the number of neurons per layer and depth is the number of layers. The final hyperparameter from each grid search is highlighted in red. The results of the final grid search can be seen in Tables \ref{tab:pnn_hpo_aucs} and \ref{tab:pnn_hpo_aucs_short}; the quoted errors are taken from the standard deviation across all $k$-folds. The \gls{AUC} values for each \mHpm point for the final chosen model are shown in Figure \ref{fig:final-auc}. 


			\begin{table}[!htb]
			  \begin{center}
					\resizebox{.65\textwidth}{!}{
			    \begin{tabular}{c | c | c | c}
			    Parameter  \\
			    \hline
			    activation function & softsign & relu & LeakyReLU \\ \hline
			    loss function & \fcolorbox{red}{white}{binary cross-entropy} & mean squared error & mean absolute error \\ \hline
			    width & 32 & & \\ \hline
			    depth & 10 & & \\ \hline
			    batch size & 1025 &  & \\ \hline
			    \end{tabular}}
			    \caption{
			      First grid, scanning over activation function and loss function. Binary cross-entropy was the chosen loss function, highlighted in red.
			    }
			    \label{Tab:taulepHPOLoss}
			  \end{center}
			\end{table}


			\begin{table}[!htb]
			  \begin{center}
					\resizebox{.55\textwidth}{!}{
			    \begin{tabular}{c | c | c | c}
			    Parameter  \\
			    \hline
			    width & 8 & 16 & 32 \\ \hline
			    depth & 3 & 5 & 10 \\ \hline
			    dropout & \fcolorbox{red}{white}{0.1} & 0.3 & \\ \hline
			    activation function & softsign & & \\ \hline
			    loss function & binary cross-entropy & & \\ \hline 
			    batch size & 1024 &  & \\ \hline
			    \end{tabular}}
			    \caption{\label{Tab:taulepHPODropout}
			      Second grid, scanning over width, depth, and dropout value. $0.1$ was chosen for the dropout value, highlighted in red.
			    }
			  \end{center}
			\end{table}

			\begin{table}[!htb]
			  \begin{center}
				\resizebox{.55\textwidth}{!}{
			    \begin{tabular}{c | c | c | c}
			    Parameter  \\
			    \hline
			    width & 32 & 64 & 128 \\ \hline
			    depth & 2 & 3 & 4 \\ \hline
			    activation function & softsign & relu & \fcolorbox{red}{white}{LeakyReLU} \\ \hline
			    dropout & 0.1 & & \\ \hline
			    batch size & 1024 &  & \\ \hline
			    loss function & binary cross-entropy  & & \\ \hline
			    \end{tabular}}
			    \caption{\label{Tab:taulepHPOActivation}
			      Third grid, scanning over activation function. LeakyReLU was chosen, highlighted in red.
			    }
			    
			  \end{center}
			\end{table}

			\begin{table}[!htb]
			  \begin{center}
				\resizebox{.55\textwidth}{!}{
			  \begin{tabular}{c | c | c | c | c}
			  Parameter  \\ 
			  \hline
			  width & 32 & 64 & 128 & \\ \hline
			  depth & 2 & 3 & 4 &  \\ \hline
			  $\alpha$ & 0.01 & \fcolorbox{red}{white}{0.05} & 0.001 & 0.005 \\ \hline
			  batch size & 1024 & & & \\ \hline
			  dropout & 0.1 & & & \\ \hline
			  activation function & LeakyReLU &  &  &  \\ \hline
			  loss function & binary cross-entropy & & &  \\ \hline
			  \end{tabular}}
			    \caption{\label{Tab:taulepHPOAlpha}
			      Fourth grid, scanning over LeakyReLU $\alpha$ value. $\alpha=0.05$ was chosen, highlighted in red.
			    }
			    
			  \end{center}
			\end{table}

			\begin{table}[!htb]
			  \begin{center}
				\resizebox{.55\textwidth}{!}{
			  \begin{tabular}{c | c | c | c | c}
			  Parameter  \\ 
			  \hline
			  width & 32 & 64 & \fcolorbox{red}{white}{128} & 256 \\
			  depth & 2 & \fcolorbox{red}{white}{3} & 4 & 5 \\
			  batch size & 1024 &  & & \\ \hline
			  dropout & 0.1 &  &  &  \\ \hline
			  activation function & LeakyReLU &  &  &  \\ \hline
			  batch size & 1024 & & & \\ \hline
			  $\alpha$ & 0.05 &  &  &  \\ \hline
			  loss function & binary cross-entropy & & &  \\ \hline
			  \end{tabular}}
			    \caption{\label{Tab:taulepHPOWidthDepth}
			      Fifth grid, scanning over network width and depth. $width=128$ and $depth=3$ was chosen, highlighted in red.
			    }
			    
			  \end{center}
			\end{table}

			\begin{table}
				\begin{center}
				\small\npdecimalsign{.}
				\nprounddigits{2}
				\resizebox{\textwidth}{!}{
				\begin{tabular}{llrrrrrrrrrrrr}
				\toprule
				width & depth &         80 &       150 &       250 &       500 &       Avg &   LowMassAvg \\
				\midrule
					128 &	3 &	0.6661	$\pm$	0.0000	&	0.8145	$\pm$	0.0000	&	0.9031	$\pm$	0.0000	&	0.9633	$\pm$	0.0000	&	0.8876	$\pm$	0.0000	&	0.8261	$\pm$	0.0968	\\
					128 &	5 &	0.6492	$\pm$	0.0000	&	0.8043	$\pm$	0.0000	&	0.9078	$\pm$	0.0000	&	0.9628	$\pm$	0.0000	&	0.8861	$\pm$	0.0000	&	0.8235	$\pm$	0.1000	\\
					128 &	4 &	0.6593	$\pm$	0.0000	&	0.8117	$\pm$	0.0000	&	0.9012	$\pm$	0.0000	&	0.9638	$\pm$	0.0000	&	0.8858	$\pm$	0.0000	&	0.8232	$\pm$	0.0994	\\
					128 &	2 &	0.6444	$\pm$	0.0000	&	0.8070	$\pm$	0.0000	&	0.9075	$\pm$	0.0000	&	0.9631	$\pm$	0.0000	&	0.8857	$\pm$	0.0000	&	0.8231	$\pm$	0.1006	\\
					64 &	4 &	0.6576	$\pm$	0.0050	&	0.8080	$\pm$	0.0013	&	0.9052	$\pm$	0.0045	&	0.9656	$\pm$	0.0016	&	0.8857	$\pm$	0.0002	&	0.8230	$\pm$	0.0994	\\
					64 &	2 &	0.6528	$\pm$	0.0066	&	0.8052	$\pm$	0.0023	&	0.9057	$\pm$	0.0032	&	0.9651	$\pm$	0.0007	&	0.8855	$\pm$	0.0004	&	0.8228	$\pm$	0.0996	\\
					64 &	5 &	0.6538	$\pm$	0.0050	&	0.8044	$\pm$	0.0019	&	0.9058	$\pm$	0.0037	&	0.9653	$\pm$	0.0014	&	0.8853	$\pm$	0.0005	&	0.8224	$\pm$	0.0997	\\
					64 &	3 &	0.6520	$\pm$	0.0067	&	0.8051	$\pm$	0.0018	&	0.9042	$\pm$	0.0044	&	0.9649	$\pm$	0.0019	&	0.8853	$\pm$	0.0011	&	0.8223	$\pm$	0.0994	\\
					256 &	5 &	0.6536	$\pm$	0.0010	&	0.8044	$\pm$	0.0033	&	0.9036	$\pm$	0.0042	&	0.9644	$\pm$	0.0022	&	0.8844	$\pm$	0.0002	&	0.8213	$\pm$	0.1003	\\
					256 &	4 &	0.6434	$\pm$	0.0000	&	0.8018	$\pm$	0.0000	&	0.9017	$\pm$	0.0000	&	0.9619	$\pm$	0.0000	&	0.8823	$\pm$	0.0000	&	0.8181	$\pm$	0.1013	\\
					32 &	3 &	0.6369	$\pm$	0.0094	&	0.7950	$\pm$	0.0041	&	0.8977	$\pm$	0.0032	&	0.9635	$\pm$	0.0022	&	0.8798	$\pm$	0.0012	&	0.8139	$\pm$	0.1031	\\
					32 &	4 &	0.6384	$\pm$	0.0037	&	0.7935	$\pm$	0.0033	&	0.8986	$\pm$	0.0037	&	0.9636	$\pm$	0.0016	&	0.8799	$\pm$	0.0009	&	0.8139	$\pm$	0.1031	\\
					32 &	2 &	0.6399	$\pm$	0.0058	&	0.7924	$\pm$	0.0024	&	0.8983	$\pm$	0.0033	&	0.9629	$\pm$	0.0023	&	0.8796	$\pm$	0.0004	&	0.8135	$\pm$	0.1023	\\
					32 &	5 &	0.6350	$\pm$	0.0077	&	0.7931	$\pm$	0.0056	&	0.8981	$\pm$	0.0022	&	0.9625	$\pm$	0.0005	&	0.8792	$\pm$	0.0011	&	0.8128	$\pm$	0.1035	\\
					256 &	2 &	0.6320	$\pm$	0.0044	&	0.7971	$\pm$	0.0000	&	0.8939	$\pm$	0.0034	&	0.9587	$\pm$	0.0018	&	0.8781	$\pm$	0.0002	&	0.8120	$\pm$	0.1023	\\
				\bottomrule
				\end{tabular}}
				\npnoround
				\caption{\label{tab:pnn_hpo_aucs}
				\glspl{AUC} of final \gls{HPO} grid. An error of 0 corresponds to only one job k-fold finishing training due to computational limits. The LowMassAvg error takes into account difference between k-folds and the associated error from the averaging calculation.
				  }
				  \end{center}
			\end{table}
			% \clearpage


			\begin{table}
				\begin{center}
				\small
				\resizebox{.5\textwidth}{!}{
				\begin{tabular}{llrrrrrrrrrrrr}
				\toprule
				width & depth &  Avg &   LowMassAvg \\
				\midrule
					128  &	3 &	0.8876	$\pm$	0.0000	&	0.8261	$\pm$	0.0968	\\
					128 &	5 &	0.8861	$\pm$	0.0000	&	0.8235	$\pm$	0.1000	\\
					128 &	4 &	0.8858	$\pm$	0.0000	&	0.8232	$\pm$	0.0994	\\
					128 &	2 &	0.8857	$\pm$	0.0000	&	0.8231	$\pm$	0.1006	\\
					64 &	4 &	0.8857	$\pm$	0.0002	&	0.8230	$\pm$	0.0994	\\
					64 &	2 &	0.8855	$\pm$	0.0004	&	0.8228	$\pm$	0.0996	\\
					64 &	5 &	0.8853	$\pm$	0.0005	&	0.8224	$\pm$	0.0997	\\
					64 &	3 &	0.8853	$\pm$	0.0011	&	0.8223	$\pm$	0.0994	\\
					256 &	5 &	0.8844	$\pm$	0.0002	&	0.8213	$\pm$	0.1003	\\
					256 &	4 &	0.8823	$\pm$	0.0000	&	0.8181	$\pm$	0.1013	\\
					32 &	3 &	0.8798	$\pm$	0.0012	&	0.8139	$\pm$	0.1031	\\
					32 &	4 &	0.8799	$\pm$	0.0009	&	0.8139	$\pm$	0.1031	\\
					32 &	2 &	0.8796	$\pm$	0.0004	&	0.8135	$\pm$	0.1023	\\
					32 &	5 &	0.8792	$\pm$	0.0011	&	0.8128	$\pm$	0.1035	\\
					256 &	2 &	0.8781	$\pm$	0.0002	&	0.8120	$\pm$	0.1023	\\
				\bottomrule
				\end{tabular}}
				\caption{\label{tab:pnn_hpo_aucs_short}
				Average \glspl{AUC} of final \gls{HPO} grid. An error of 0 corresponds to only one job k-fold finishing training due to computational limits.
				  }
			  \end{center}
			\end{table}


			\clearpage
			\begin{figure}
			  \centering
			  \includegraphics[width=\textwidth,keepaspectratio=true]{chapters/chapter6_Hplus/images/AUC_Plots/model_GB_1024_channel_taulep_mass_80to3000_ntracks_1_nfolds_5_fold_4_nvars_19_batch_size_1024_epochs_1000_dense_layer_size_128_activation_function_LeakyRelu_depth_3_loss_binary_crossentropy_dropout_0.1_alpha_0.05.eps}\\
			  \caption{Final model \gls{AUC} for each mass point. Individual points correspond to the \gls{AUC} average over 5 kfolds. }
			  \label{fig:final-auc}
			\end{figure}

			The final model was chosen to have 128 neurons per layer with three layers, with the binary cross-entropy chosen as the loss function, a dropout of 0.1, LeakyReLU as the activation function with $\alpha=0.05$. The LeakyReLU activation function is depicted in Figure \ref{fig:leaky-relu}, where the $\alpha$ value is the slope of the negative portion. Allowing negative weight values prevents neurons from becoming deactivated prematurely.

			\begin{figure}
				\centering
				\includegraphics[width=.4\textwidth,keepaspectratio=true]{chapters/chapter6_HPlus/images/Activation_prelu.svg.png}
				\caption{LeakyReLU activation function. The associated hyperparameter $\alpha$ is the slope of the negative portion of the function.}
				\label{fig:leaky-relu}
			\end{figure}

	\section{Systematic Uncertainties}\label{sec:systs}
		Systematic uncertainties have a variety of sources and are discussed here. Detector-related systematic uncertainties from the reconstruction and identification of leptons and \tauhad objects \cite{tau-calibration}, simulation of the electron and muon triggers, reconstruction of \Etm, and energy/momentum scale and resolution of all physics objects such as $\tau$ leptons \cite{tau-calibration}, jets \cite{jet-calibration}, electrons \cite{egamma-calibration}, \Etm \cite{met-perf}, and muons \cite{muon-calibration} are studied by varying selection cuts by $\pm 1$ standard deviation. The difference in event yields is then taken as a systematic error and summed in quadrature with all other sources of error to give the final quoted errors. Systematic errors resulting in an upward fluctuation are kept separate from downward fluctuations. The effect of the main sources of uncertainties on the event yield for \ttbar and an arbitrary mass point of $\mHpm=200$ GeV are shown in Table \ref{tab:sys_ttbar_signal} for all \glspl{SR}. Jet systematic uncertainties arising from reconstruction, identification, flavor composition, resolution account for the largest contribution. Systematic uncertainties arising from the data-driven fake factor method are shown in Table \ref{tab:FF-syst-uncert}. The uncertainty listed as ``Fake factors: True \tauhad in the anti-\tauhad CR'' is a conservative uncertainty of 50\% applied to account for amount of \tauhad candidates that are subtracted when computing $N^{anti-\tau}_{fakes}$. Table \ref{tab:sys_combined_ttbar_signal} shows systematic uncertainties combined by their source. The combined fake factor systematic uncertainty is the largest contributor, followed by the jet systematics listed above. Theoretical uncertainties for signal and \ttbar background were considered in the last publication; at the time of writing this dissertation the simulations are being produced and therefore are not included. In the previous result, \Hpm signal modelling systematic errors were found to have an impact on expected limits of 2.5\% for $\mHpm=170$ GeV and 6.4\% for $\mHpm=1000$ GeV \cite{hpm-previous}.
		
		\begin{table}
			\begin{center}
			\resizebox{\textwidth}{!}{
			\begin{tabular}{| l | c | c | c | c | c | c |}
			\hline
			Source                     							&  	\multicolumn{6}{c|}{Impact on the expected event yield (\%)}  \\ \cline{2-7}
						                     							&  	\multicolumn{2}{c|}{\taujets} 																																			&  	\multicolumn{2}{c|}{\tauel} 																																			&  	\multicolumn{2}{c|}{\taumu} 		\\ \cline{2-7}
			                             						& 	\ttbar   																				&  \Hpm 200 GeV  																		& 	\ttbar   																			&  	\Hpm 200 GeV 																		& 	\ttbar   																				&  	\Hpm 200 GeV 																			\\ \hline \hline
			\tauhad reconstruction efficiency 			&  	$\pm$1.24																				&  $\pm$1.22 																				&  	$\pm$1.23 																		&   \begin{tabular}{c}+1.22 \\ -1.23 \end{tabular}	&  	$\pm$1.23 																			& 	$\pm$1.22																					\\ \cline{2-7}
			\tauhad-id  														&  	$\pm$1.79 																			&  $\pm$0.52 																				&  	$\pm$1.40 																		&   $\pm$0.50																				&  	$\pm$1.40 																			& 	$\pm$0.48																					\\ \cline{2-7}
			\tauhad energy scale 										&  	\begin{tabular}{c}+2.53 \\ -2.80 \end{tabular}  &   \begin{tabular}{c}+2.00 \\ -1.66 \end{tabular}	& \begin{tabular}{c}+1.60 \\ -1.44 \end{tabular} 	&   \begin{tabular}{c}+1.28 \\ -1.66 \end{tabular} 	&  	\begin{tabular}{c}+1.53 \\ -1.39 \end{tabular} 	& 	\begin{tabular}{c}+1.72 \\ -1.46 \end{tabular}		\\ \hline
			\tauhad energy scale (detector) 				&  	\begin{tabular}{c}+1.96 \\ -1.55 \end{tabular}  &   \begin{tabular}{c}+1.64 \\ -1.49 \end{tabular}	& \begin{tabular}{c}+0.23 \\ -0.21 \end{tabular} 	&   \begin{tabular}{c}+1.15 \\ -1.08 \end{tabular} 	&  	\begin{tabular}{c}+0.16 \\ -0.55 \end{tabular} 	& 	\begin{tabular}{c}+0.49 \\ -1.5 \end{tabular}			\\ \cline{2-7}
			\tauhad energy scale (in-situ)   				&  	\begin{tabular}{c}+1.44 \\ -1.43 \end{tabular}  &   \begin{tabular}{c}+0.22 \\ -0.74 \end{tabular}	& \begin{tabular}{c}+1.17 \\ -1.20 \end{tabular} 	&   \begin{tabular}{c}+0.74 \\ -0.63 \end{tabular} 	&  	\begin{tabular}{c}+1.14 \\ -1.15 \end{tabular} 	& 	\begin{tabular}{c}+0.54 \\ -0.37 \end{tabular}		\\ \cline{2-7}
			\tauhad energy scale (model)    				&  	\begin{tabular}{c}+0.56 \\ -0.61 \end{tabular} 	&   -0.06																						& \begin{tabular}{c}+0.23 \\ -0.21 \end{tabular} 	&   \begin{tabular}{c}+1.15 \\ -1.08 \end{tabular}  &  	\begin{tabular}{c}+0.16 \\ -0.55 \end{tabular} 	& 	\begin{tabular}{c}+0.49 \\ -1.50 \end{tabular}		\\ \cline{2-7}
			\tauhad energy scale (physics list)			&  	\begin{tabular}{c}+1.27 \\ -1.26 \end{tabular} 	&   -0.72																						& \begin{tabular}{c}+0.74 \\ -0.65 \end{tabular} 	&   \begin{tabular}{c}+0.67 \\ -0.25 \end{tabular}  &  	\begin{tabular}{c}+0.72 \\ -0.63 \end{tabular} 	& 	\begin{tabular}{c}+0.83 \\ -0.60 \end{tabular}		\\ \hline
			jet uncertainties												&  	\begin{tabular}{c}+7.38 \\ -8.39 \end{tabular}	&   \begin{tabular}{c}+6.51 \\ -9.06 \end{tabular} 	& \begin{tabular}{c}+3.41 \\ -3.31 \end{tabular} 	&   \begin{tabular}{c}+4.49 \\ -2.78 \end{tabular} 	&  	\begin{tabular}{c}+3.18 \\ -3.24 \end{tabular} 	& 	\begin{tabular}{c}+3.67 \\ -2.96 \end{tabular}		\\ 
			\Etm soft term scale/resolution 				&  	\begin{tabular}{c}+1.31 \\ -1.12 \end{tabular} 	&   \begin{tabular}{c}+1.15 \\ -1.49 \end{tabular}	& \begin{tabular}{c}+0.29 \\ -0.24 \end{tabular} 	&   \begin{tabular}{c}+0.88 \\ -0.34 \end{tabular} 	&  	\begin{tabular}{c}+0.30 \\ -0.23 \end{tabular} 	& 	\begin{tabular}{c}+0.21 \\ -0.11 \end{tabular}		\\  
			trigger                   &   \begin{tabular}{c}+1.23 \\ -1.61 \end{tabular}  &   0                                               & $\pm$0.03                                       &   0                                               &   \begin{tabular}{c}+0.55 \\ -0.56 \end{tabular}  &   $\pm$0.56                                           \\ \hline
	        $e$-id  																&  	0 																							&   0 																							& $\pm$0.71 																			&   $\pm$0.73 																			&  	0 																							& 	0																									\\ \hline
			$\mu$-id/reconstruction/isolation  			&  	0 																							&   0 																							& \begin{tabular}{c}0 \\ -0.01 \end{tabular} 	&   \begin{tabular}{c}0 \\ -0.11 \end{tabular} 	&  	\begin{tabular}{c}+0.97 \\ -1.40 \end{tabular} 	& 	\begin{tabular}{c}+1.00 \\ -2,94 \end{tabular}		\\
			$\mu$ MS 																&  	0 																							&   0 																							& 0 																							&   0 																							&  	\begin{tabular}{c}+0.09 \\ -0.12 \end{tabular} 	& 	\begin{tabular}{c}+0.40 \\ -0.34 \end{tabular}				\\
			\hline
			\end{tabular}}
			\caption{\label{tab:sys_ttbar_signal}
			Effect of the main systematic uncertainties on the expected event yield for \ttbar and signal events ($\mHpm= 200$ GeV) passing the nominal event selection of the three \glspl{SR}. The three components of the \tauhad energy scale uncertainty are shown in the table. Impacts are shown in percent change with respect to the nominal \gls{SR} selections.
			}
			\end{center}
		\end{table}

		\begin{table}[!htb]
		  \begin{center}
				\resizebox{\textwidth}{!}{
		    \begin{tabular}{l|c|c||c|c||}
		      &  \multicolumn{2}{c||}{\taujets}  & \multicolumn{2}{c||}{\taulep} \\
		      \hline
		      Source of uncertainty & Effect on yield & Shape & Effect on yield & Shape \\
		      \hline\hline
		      Fake factors: statistical uncertainties & 3.9$\%$ & \textcolor{red}{\ding{55}} & 3.2$\%$ & \textcolor{red}{\ding{55}}\\\hline
		      Fake factors: True \tauhad in the anti-\tauhad CR & \begin{tabular}{c}+3.4\% \\-3.2\%\end{tabular} & \textcolor{red}{\ding{55}} & \begin{tabular}{c}+4\% \\-4.3\%\end{tabular} & \textcolor{red}{\ding{55}}\\\hline
		      Fake factors: tau RNN Identification SF & 2.7$\%$ & \textcolor{red}{\ding{51}} & 2.7$\%$ & \textcolor{red}{\ding{51}}\\\hline
		      Fake factors: $\alpha_\text{MJ}$ uncertainty & 3.6\% & \textcolor{red}{\ding{55}} & 1.9\% & \textcolor{red}{\ding{55}}\\\hline
		      Fake factors: heavy flavor jet fraction & 6$\%$ & \textcolor{red}{\ding{51}} & 5.53$\%$ & \textcolor{red}{\ding{51}}\\
		    \end{tabular}}
		  \end{center}
		  \caption{\label{tab:FF-syst-uncert}
		    Effect on the shape variation and the yields of systematic uncertainties associated with the data-driven fake
		    factor method, used to estimate the $j \to \tau$ background in the
		   \taujets and \taulep channel.
		    }
		\end{table}

		\begin{table}[!htb]
        \begin{center}
        \resizebox{\textwidth}{!}{
        \begin{tabular}{| l | c | c | c | c | c | c |}
        \hline
        Source                                  &   \multicolumn{6}{|c|}{Impact on the expected event yield (\%)}  \\ \cline{2-7}
                                                &   \multicolumn{2}{|c|}{\taujets}                                                                       &   \multicolumn{2}{c|}{\tauel}                                                                       &   \multicolumn{2}{c|}{\taumu}     \\ \cline{2-7}
                                                &   \ttbar                                          &  \Hpm 200 GeV                                     &   \ttbar                                        &   \Hpm 200 GeV                                    &   \ttbar                                          &   \Hpm 200 GeV                                      \\ \hline \hline
        % \tauhad reconstruction efficiency       &   $\pm$1.24                                       &  $\pm$1.22                                        &   $\pm$1.23                                     &   \begin{tabular}{c}+1.22 \\ -1.23 \end{tabular}  &   $\pm$1.23                                       &   $\pm$1.22                                         \\ \hline
        % \tauhad-id                              &   $\pm$1.79                                       &  $\pm$0.52                                        &   $\pm$1.40                                     &   $\pm$0.50                                       &   $\pm$1.40                                       &   $\pm$0.48                                         \\ \hline
        % \tauhad energy scale                    &   \begin{tabular}{c}+2.53 \\ -2.80 \end{tabular}  &   \begin{tabular}{c}+2.00 \\ -1.66 \end{tabular}  & \begin{tabular}{c}+1.60 \\ -1.44 \end{tabular}  &   \begin{tabular}{c}+1.28 \\ -1.66 \end{tabular}  &   \begin{tabular}{c}+1.53 \\ -1.39 \end{tabular}  &   \begin{tabular}{c}+1.72 \\ -1.46 \end{tabular}    \\ \hline
        % \tauhad energy scale (detector)         &   \begin{tabular}{c}+1.96 \\ -1.55 \end{tabular}  &   \begin{tabular}{c}+1.64 \\ -1.49 \end{tabular}  & \begin{tabular}{c}+0.23 \\ -0.21 \end{tabular}  &   \begin{tabular}{c}+1.15 \\ -1.08 \end{tabular}  &   \begin{tabular}{c}+0.16 \\ -0.55 \end{tabular}  &   \begin{tabular}{c}+0.49 \\ -1.5 \end{tabular}     \\ \hline
        % \tauhad energy scale (in-situ)          &   \begin{tabular}{c}+1.44 \\ -1.43 \end{tabular}  &   \begin{tabular}{c}+0.22 \\ -0.74 \end{tabular}  & \begin{tabular}{c}+1.17 \\ -1.20 \end{tabular}  &   \begin{tabular}{c}+0.74 \\ -0.63 \end{tabular}  &   \begin{tabular}{c}+1.14 \\ -1.15 \end{tabular}  &   \begin{tabular}{c}+0.54 \\ -0.37 \end{tabular}    \\ \hline
        % \tauhad energy scale (model)            &   \begin{tabular}{c}+0.56 \\ -0.61 \end{tabular}  &   -0.06                                           & \begin{tabular}{c}+0.23 \\ -0.21 \end{tabular}  &   \begin{tabular}{c}+1.15 \\ -1.08 \end{tabular}  &   \begin{tabular}{c}+0.16 \\ -0.55 \end{tabular}  &   \begin{tabular}{c}+0.49 \\ -1.50 \end{tabular}    \\ \hline
        % \tauhad energy scale (physics list)     &   \begin{tabular}{c}+1.27 \\ -1.26 \end{tabular}  &   -0.72                                           & \begin{tabular}{c}+0.74 \\ -0.65 \end{tabular}  &   \begin{tabular}{c}+0.67 \\ -0.25 \end{tabular}  &   \begin{tabular}{c}+0.72 \\ -0.63 \end{tabular}  &   \begin{tabular}{c}+0.83 \\ -0.60 \end{tabular}    \\ \hline
        Fake factor uncertainties               &   \begin{tabular}{c}+9.11 \\ -9.04 \end{tabular}  &   \begin{tabular}{c}+9.11 \\ -9.04 \end{tabular}  & \begin{tabular}{c}+8.23 \\ -8.34 \end{tabular}  &   \begin{tabular}{c}+8.23 \\ -8.34 \end{tabular}  &   \begin{tabular}{c}+8.23 \\ -8.34 \end{tabular}  &   \begin{tabular}{c}+8.23 \\ -8.34\end{tabular}    \\ \hline
        jet uncertainties                       &   \begin{tabular}{c}+7.38 \\ -8.39 \end{tabular}  &   \begin{tabular}{c}+6.51 \\ -9.06 \end{tabular}  & \begin{tabular}{c}+3.41 \\ -3.31 \end{tabular}  &   \begin{tabular}{c}+4.49 \\ -2.78 \end{tabular}  &   \begin{tabular}{c}+3.18 \\ -3.24 \end{tabular}  &   \begin{tabular}{c}+3.67 \\ -2.96 \end{tabular}    \\ \hline
        $\tau$ uncertainties                    &   $\pm4.36$                                       &   \begin{tabular}{c}+2.91 \\ -2.80 \end{tabular}  & \begin{tabular}{c}+2.84 \\ -2.74 \end{tabular}  &   \begin{tabular}{c}+2.65 \\ -2.70 \end{tabular}  &   \begin{tabular}{c}+2.77 \\ -2.78 \end{tabular}  &   \begin{tabular}{c}+2.58 \\ -2.97 \end{tabular}    \\ \hline
        \Etm uncertainties                      &   \begin{tabular}{c}+1.31 \\ -1.12 \end{tabular}  &   \begin{tabular}{c}+1.15 \\ -1.49 \end{tabular}  & \begin{tabular}{c}+0.29 \\ -0.24 \end{tabular}  &   \begin{tabular}{c}+0.88 \\ -0.34 \end{tabular}  &   \begin{tabular}{c}+0.30 \\ -0.23 \end{tabular}  &   \begin{tabular}{c}+0.21 \\ -0.11 \end{tabular}    \\ \hline
        trigger uncertainties                   &   \begin{tabular}{c}+1.23 \\ -1.61 \end{tabular}  &   0                                               & $\pm$0.03                                       &   0                                               &   \begin{tabular}{c}+0.55 \\ -0.56 \end{tabular}  &   $\pm$0.56                                           \\ \hline
        $e$ uncertainties                       &   0                                               &   0                                               & $\pm$0.71                                       &   $\pm$0.73                                       &   0                                               &   0                                                 \\ \hline
        $\mu$ uncertainties                     &   0                                               &   0                                               & -0.01                                           &   -0.11                                           &   \begin{tabular}{c}+0.97 \\ -1.41 \end{tabular}  &   \begin{tabular}{c}+1.08 \\ -2,96 \end{tabular}    \\ \hline
        % $\mu$ MS                                &   0                                               &   0                                               & 0                                               &   0                                               &   \begin{tabular}{c}+0.09 \\ -0.12 \end{tabular}  &   \begin{tabular}{c}+0.40 \\ -0.34 \end{tabular}        \\
        \hline
        \end{tabular}}
        \caption{\label{tab:sys_combined_ttbar_signal}
        Effect of the combined main systematic uncertainties on the expected event yield for \ttbar and signal events ($\mHpm= 200$ GeV) passing the nominal event selection of the three \glspl{SR}. The three components of the \tauhad energy scale uncertainty are shown in the table. Impacts are shown in percent change with respect to the nominal \gls{SR} selections.
        }
        \end{center}
      \end{table}

	\section{Expected Results}\label{sec:results}
		The expected event yields for backgrounds and signal\footnote{At the time of writing, the analysis is still blinded so data is not included.} are summarized in Table \ref{tab:event_yields_SR_taujets} (\taujets) and Table \ref{tab:event_yields_SR_taulep} (\taulep).

		\begin{table}
			\begin{center}
			\caption{\label{tab:event_yields_SR_taujets} Expected event yields for the backgrounds and a hypothetical \Hpm signal after applying all \taujets selection criteria, and comparison with \LUMI of data. The values shown for the signal assuming a charged Higgs boson mass of 170 GeV and 1000 GeV, with a cross-section times branching fraction $\sigma(\pp \to tb\Hpm) \times \mathrm{\cal{B}}(\Hpm \to \tau\nu)$ corresponding to $\tanb=40$ in the hMSSM benchmark scenario. Statistical and systematic uncertainties are quoted, respectively.
			}
			\resizebox{\textwidth}{!}{
			\begin{tabular}{l|r}
			Sample & Event yields $\tauhad$+jets          \\
			\hline
			\ttbar & $\phantom{}18443\phantom{.0}\pm\phantom{0}48\phantom{.0}\phantom{0}\begin{tabular}{c}+1545 \\-1697\end{tabular}$    \\ \hline
			Single-top-quark  & $\phantom{0}2284\phantom{.0}\pm\phantom{0}17\phantom{.0}\phantom{0}\begin{tabular}{c}+184 \\-207\end{tabular}$ \\ \hline
			$W \to \tau\nu$  & $\phantom{0}1979\phantom{.0}\pm\phantom{0}23\phantom{.0}\phantom{0}\begin{tabular}{c}+179 \\-229\end{tabular}$  \\ \hline
			$Z \to \tau\tau$  & $\phantom{00}242\phantom{.0}\pm\phantom{00}5\phantom{.0}\phantom{0}\begin{tabular}{c}+24 \\-32\end{tabular}$   \\ \hline
			Diboson ($WW, WZ, ZZ$) & $\phantom{00}133\phantom{.0}\pm\phantom{00}4\phantom{.0}\phantom{0}\begin{tabular}{c}+9 \\-12\end{tabular}$  \\ \hline
			Misidentified $e,\,\mu \to \tauhad$   & $\phantom{00}328\phantom{.0}\pm\phantom{00}6\phantom{.0}\phantom{0}\begin{tabular}{c}+25 \\-34\end{tabular}$  \\ \hline
			Misidentified $\mbox{jet} \to \tauhad$ & $\phantom{0}2506\phantom{.0}\pm\phantom{0}17\phantom{.0}\phantom{0}\begin{tabular}{c}+130 \\-133\end{tabular}$ \\ \hline
			\hline
			All backgrounds   & $25917\phantom{.0}\pm\phantom{0}59\phantom{.0}\phantom{0}\begin{tabular}{c}+1572 \\-1730\end{tabular}$  \\
			\hline
			\Hpm $\phantom{0}$(170 GeV), hMSSM $\tanb=40$ & $\phantom{0}1075\phantom{.0}\pm\phantom{0}9\phantom{.0}\phantom{0}\begin{tabular}{c}+82 \\-79\end{tabular}$  \\
			\Hpm (1000 GeV), hMSSM $\tanb=40$ & $\phantom{000}12910\phantom{0.}\pm\phantom{00}59\phantom{0.}\phantom{0}\begin{tabular}{c}+784 \\-720\end{tabular}$  \\
			\hline
			\end{tabular}}
			\end{center}
		\end{table}

		\begin{table}
			\begin{center}
			\caption{\label{tab:event_yields_SR_taulep} Expected event yields for the backgrounds and a hypothetical \Hpm signal after applying all \taulep selection criteria, and comparison with \LUMI of data. The values shown for the signal assuming a charged Higgs boson mass of 170 GeV and 1000 GeV, with a cross-section times branching fraction $\sigma(\pp \to tb\Hpm) \times \mathrm{\cal{B}}(\Hpm \to \tau\nu)$ corresponding to $\tanb=40$ in the hMSSM benchmark scenario. Statistical and systematic uncertainties are quoted, respectively.
			}
			\resizebox{\textwidth}{!}{
			\begin{tabular}{l|r|r}
			Sample &   Event yields \tauel  & Event yields \taumu \\
			\hline
			$t\bar{t}$ & $43813\phantom{.0}\pm\phantom{0}76\phantom{.0}\phantom{0}\begin{tabular}{c}+1749 \\-1833\end{tabular}$ & $44486\phantom{.0}\pm\phantom{0}75\phantom{.0}\phantom{0}\begin{tabular}{c}+1811 \\-1907\end{tabular}$  \\ \hline
			Single-top-quark  & $\phantom{0}3260\phantom{.0}\pm\phantom{0}20\phantom{.0}\phantom{0}\begin{tabular}{c}+124 \\-134\end{tabular}$ & $\phantom{0}3873\phantom{.0}\pm\phantom{0}22\phantom{.0}\phantom{0}\begin{tabular}{c}+158 \\-165\end{tabular}$  \\ \hline
			$W \to \tau\nu$  & $\phantom{0000}2.41\phantom{.0}\pm\phantom{00}0.56\phantom{.0}\phantom{0}\begin{tabular}{c}+0.22 \\-2.15\end{tabular}$ & $\phantom{0000}0.07\phantom{.0}\pm\phantom{00}0.12\phantom{.0}\phantom{0}\begin{tabular}{c}+0.08 \\-0.16\end{tabular}$  \\ \hline
			$Z \to \tau\tau$  & $\phantom{00}913\phantom{.0}\pm\phantom{0}20\phantom{.0}\phantom{0}\begin{tabular}{c}+64 \\-149\end{tabular}$  & $\phantom{00}845\phantom{.0}\pm\phantom{0}22\phantom{.0}\phantom{0}\begin{tabular}{c}+88 \\-111\end{tabular}$ \\ \hline
			Diboson ($WW, WZ, ZZ$) &  $\phantom{000}72.64\phantom{.0}\pm\phantom{00}1.52\phantom{.0}\phantom{0}\begin{tabular}{c}+5.25 \\-3.91\end{tabular}$ & $\phantom{000}80.81\phantom{.0}\pm\phantom{00}1.53\phantom{.0}\phantom{0}\begin{tabular}{c}+5.40 \\-6.45\end{tabular}$ \\ \hline
			Misidentified $e,\,\mu \to \tauhad$   &   $\phantom{0}1083\phantom{.0}\pm\phantom{0}24\phantom{.0}\phantom{0}\begin{tabular}{c}+41 \\-73\end{tabular}$ & $\phantom{0}1060\phantom{.0}\pm\phantom{0}15\phantom{.0}\phantom{0}\begin{tabular}{c}+43 \\-70\end{tabular}$ \\ \hline
			Misidentified $\mbox{jet} \to \tauhad$ &   $\phantom{0}8662\phantom{.0}\pm\phantom{0}37\phantom{.0}\phantom{0}\begin{tabular}{c}+450 \\-470\end{tabular}$ & $\phantom{0}8426\phantom{.0}\pm\phantom{0}37\phantom{.0}\phantom{0}\begin{tabular}{c}+440 \\-459\end{tabular}$ \\ \hline
			\hline
			All backgrounds   &  $57809\phantom{.0}\pm\phantom{0} 93\phantom{.0}\phantom{0}\begin{tabular}{c}+1812 \\-1846\end{tabular}$ & $58773\phantom{.0}\pm\phantom{0} 90\phantom{.0}\phantom{0}\begin{tabular}{c}+1873 \\-1970\end{tabular}$ \\
			\hline
			\Hpm $\phantom{0}$(170 GeV), hMSSM $\tanb=40$ & $\phantom{0}598\phantom{.0}\pm\phantom{0}6\phantom{.0}\phantom{0}\begin{tabular}{c}+20 \\-22\end{tabular}$ & $\phantom{0}702\phantom{.0}\pm\phantom{0}6\phantom{.0}\phantom{0}\begin{tabular}{c}+22 \\-16\end{tabular}$ \\
			\Hpm (1000 GeV), hMSSM $\tanb=40$ &   $\phantom{0000}938\phantom{0.}\pm\phantom{00}13\phantom{0.}\phantom{0}\begin{tabular}{c}+48 \\-37\end{tabular}$ & $\phantom{0000}1024\phantom{0.}\pm\phantom{00}13\phantom{0.}\phantom{0}\begin{tabular}{c}+48 \\-57\end{tabular}$ \\
			\hline
			\end{tabular}}
			\end{center}
		\end{table}

		The test statistic $\tilde{q}_{\mu}$ \cite{test-statistic} is used to test the agreement of the data with the background-only and signal+background hypotheses. The test statistic is based on a profile likelihood ratio where the binned likelihood function $\cal{L}(\mu,\theta)$ is constructed as the product of Poisson probability terms over all bins and regions. The likelihood ratio is the ratio between the conditional maximum-likelihood estimator of the nuisance parameters, $\theta$, for a given signal hypothesis $\mu$ and the unconditional maximum-likelihood estimator for $\mu$ and the nuisance parameters. $\tilde{q}_{\mu}$ is defined as:
		\begin{equation}
		\tilde{q}_{\mu} = \left\{
		\begin{array}{ll}
		-2\ln\frac{{\cal L}(\mu, \hat{\hat{\theta}}(\mu))}{{\cal L}(0, \hat{\hat{\theta}}(0))}, & \hat{\mu} < 0 \\
		-2\ln\frac{{\cal L}(\mu, \hat{\hat{\theta}}(\mu))}{{\cal L}(\hat{\mu}, \hat{\theta})}, & 0 \leq \hat{\mu} \leq \mu \\
		0 & \hat{\mu} > \mu 
		\end{array}
		\right.
		\end{equation}
		Here, $\hat{\mu}$ and $\hat{\theta}$ are the values of the parameters that maximize the likelihood function. $\hat{\hat{\theta}}(\mu)$ corresponds to the values of the nuisance parameters that maximize the likelihood function for a given signal strength $\mu$.

		The fit is performed on the \gls{PNN} score distributions in the three \glspl{SR}, \taujets, \tauel, \taumu, and the dilepton-btag \gls{CR} which is enriched in the dominant \ttbar background. Pre-fit \gls{PNN} score distributions are shown in Figures \ref{fig:taujets_SR_PNNscores_body-1} - \ref{fig:taulepPNNscoreSR2_body-2}. At the time of writing this dissertation, the analysis is still blinded. Assuming the fit agrees with the background-only hypothesis expected limits of $\sigma(\pp \to tb\Hpm)\times \mathrm{\cal{B}}(\Hpm \to \tau \nu)$ are calculated. Exclusion limits are set at the 95\% confidence level (CL) using the $\mathrm{CL}_s$ procedure \cite{CL-setting}. The expected exclusion limits on $\sigma(\pp \to tb\Hpm)\times \mathrm{\cal{B}}(\Hpm \to \tau \nu)$ can be seen in Figure \ref{fig:expected-limits} compared to the previous 36.1 \ifb result. In all three subchannels an improvement across the entire \mHpm range can be seen. The result of this search is expected to improve the limits by a factor of $\gtrsim 3 \times$ depending on the \mHpm value. The \tauel and \taumu \glspl{SR} outperform the \taujets \gls{SR} at low \mHpm while the \taujets \gls{SR} excels at high mass values. In all three \glspl{SR} the limits turn upwards between 2500 GeV and 3000 GeV; this is due to the decreased signal acceptance shown in Figure \ref{fig:signal-acceptance}. At the time of writing the combination of the \taulep and \taujets subchannels is underway. As in the previous result \cite{hpm-previous} the combined limit will be extrapolated to set limits on \tanb.

		\begin{figure}
		  \centering
			\subfloat[\label{fig:taujets_SR_PNNscores_body_a}]{\includegraphics[width=0.45\linewidth]{chapters/chapter6_HPlus/images/taujets/clf_score_GB200_mass_80to80_SR_TAUJET.png}}
			\subfloat[\label{fig:taujets_SR_PNNscores_body_b}]{\includegraphics[width=0.45\linewidth]{chapters/chapter6_HPlus/images/taujets/clf_score_GB200_mass_130to130_SR_TAUJET.png}} \\
			\subfloat[\label{fig:taujets_SR_PNNscores_body_c}]{\includegraphics[width=0.45\linewidth]{chapters/chapter6_HPlus/images/taujets/clf_score_GB200_mass_160to160_SR_TAUJET.png}}
			\subfloat[\label{fig:taujets_SR_PNNscores_body_d}]{\includegraphics[width=0.45\linewidth]{chapters/chapter6_HPlus/images/taujets/clf_score_GB200_mass_200to200_SR_TAUJET.png}} \\
			% \subfloat[\label{fig:taujets_SR_PNNscores_body_e}]{\includegraphics[width=0.45\linewidth]{chapters/chapter6_HPlus/images/taujets/clf_score_GB200_mass_500to500_SR_TAUJET.png}}
			% \subfloat[\label{fig:taujets_SR_PNNscores_body_f}]{\includegraphics[width=0.45\linewidth]{chapters/chapter6_HPlus/images/taujets/clf_score_GB200_mass_3000to3000_SR_TAUJET.png}}
			  \caption{\label{fig:taujets_SR_PNNscores_body-1} \gls{PNN} score distributions in the
			signal region of the \taujets channel, for chosen charged Higgs boson mass parameters.
			The lower panel of each plot shows the ratio of data to the \acrshort{SM} background prediction. The uncertainty bands include all statistical and systematic uncertainties. 
			The normalization of the signal (shown for illustration) corresponds to the integral of the background.}
		\end{figure}

		\begin{figure}
		  \centering
			% \subfloat[\label{fig:taujets_SR_PNNscores_body_a}]{\includegraphics[width=0.45\linewidth]{chapters/chapter6_HPlus/images/taujets/clf_score_GB200_mass_80to80_SR_TAUJET.png}}
			% \subfloat[\label{fig:taujets_SR_PNNscores_body_b}]{\includegraphics[width=0.45\linewidth]{chapters/chapter6_HPlus/images/taujets/clf_score_GB200_mass_130to130_SR_TAUJET.png}} \\
			% \subfloat[\label{fig:taujets_SR_PNNscores_body_c}]{\includegraphics[width=0.45\linewidth]{chapters/chapter6_HPlus/images/taujets/clf_score_GB200_mass_160to160_SR_TAUJET.png}}
			% \subfloat[\label{fig:taujets_SR_PNNscores_body_d}]{\includegraphics[width=0.45\linewidth]{chapters/chapter6_HPlus/images/taujets/clf_score_GB200_mass_200to200_SR_TAUJET.png}} \\
			\subfloat[\label{fig:taujets_SR_PNNscores_body_e}]{\includegraphics[width=0.45\linewidth]{chapters/chapter6_HPlus/images/taujets/clf_score_GB200_mass_500to500_SR_TAUJET.png}}
			\subfloat[\label{fig:taujets_SR_PNNscores_body_f}]{\includegraphics[width=0.45\linewidth]{chapters/chapter6_HPlus/images/taujets/clf_score_GB200_mass_3000to3000_SR_TAUJET.png}}
			  \caption{\label{fig:taujets_SR_PNNscores_body-2} \gls{PNN} score distributions in the
			signal region of the \taujets channel, for chosen charged Higgs boson mass parameters.
			The lower panel of each plot shows the ratio of data to the \acrshort{SM} background prediction. The uncertainty bands include all statistical and systematic uncertainties. 
			The normalization of the signal (shown for illustration) corresponds to the integral of the background.}
		\end{figure}

		\begin{figure}
			\centering
			\subfloat[\label{fig:taulepPNNscoreSR1_body_a}]{\includegraphics[width=0.45\linewidth]{chapters/chapter6_HPlus/images/taulep/clf_score_GB200_mass_80to80_SR_TAUEL.png}}
			\subfloat[\label{fig:taulepPNNscoreSR1_body_b}]{\includegraphics[width=0.45\linewidth]{chapters/chapter6_HPlus/images/taulep/clf_score_GB200_mass_130to130_SR_TAUEL.png}} \\
			\subfloat[\label{fig:taulepPNNscoreSR1_body_c}]{\includegraphics[width=0.45\linewidth]{chapters/chapter6_HPlus/images/taulep/clf_score_GB200_mass_160to160_SR_TAUEL.png}}
			\subfloat[\label{fig:taulepPNNscoreSR1_body_d}]{\includegraphics[width=0.45\linewidth]{chapters/chapter6_HPlus/images/taulep/clf_score_GB200_mass_200to200_SR_TAUEL.png}} \\
			% \subfloat[\label{fig:taulepPNNscoreSR1_body_e}]{\includegraphics[width=0.45\linewidth]{chapters/chapter6_HPlus/images/taulep/clf_score_GB200_mass_500to500_SR_TAUEL.png}}
			% \subfloat[\label{fig:taulepPNNscoreSR1_body_f}]{\includegraphics[width=0.45\linewidth]{chapters/chapter6_HPlus/images/taulep/clf_score_GB200_mass_3000to3000_SR_TAUEL.png}}
			\caption{\label{fig:taulepPNNscoreSR1_body-1} \gls{PNN} score distributions in the
			signal region of the \tauel sub-channel, for the chosen charged Higgs boson mass parameters.
			The lower panel of each plot shows the ratio of data to the \acrshort{SM} background prediction. The uncertainty bands include all statistical and systematic uncertainties.
			The normalization of the signal (shown for illustration) corresponds to the integral of the background.}
		\end{figure}

		\begin{figure}
			\centering
			% \subfloat[\label{fig:taulepPNNscoreSR1_body_a}]{\includegraphics[width=0.45\linewidth]{chapters/chapter6_HPlus/images/taulep/clf_score_GB200_mass_80to80_SR_TAUEL.png}}
			% \subfloat[\label{fig:taulepPNNscoreSR1_body_b}]{\includegraphics[width=0.45\linewidth]{chapters/chapter6_HPlus/images/taulep/clf_score_GB200_mass_130to130_SR_TAUEL.png}} \\
			% \subfloat[\label{fig:taulepPNNscoreSR1_body_c}]{\includegraphics[width=0.45\linewidth]{chapters/chapter6_HPlus/images/taulep/clf_score_GB200_mass_160to160_SR_TAUEL.png}}
			% \subfloat[\label{fig:taulepPNNscoreSR1_body_d}]{\includegraphics[width=0.45\linewidth]{chapters/chapter6_HPlus/images/taulep/clf_score_GB200_mass_200to200_SR_TAUEL.png}} \\
			\subfloat[\label{fig:taulepPNNscoreSR1_body_e}]{\includegraphics[width=0.45\linewidth]{chapters/chapter6_HPlus/images/taulep/clf_score_GB200_mass_500to500_SR_TAUEL.png}}
			\subfloat[\label{fig:taulepPNNscoreSR1_body_f}]{\includegraphics[width=0.45\linewidth]{chapters/chapter6_HPlus/images/taulep/clf_score_GB200_mass_3000to3000_SR_TAUEL.png}}
			\caption{\label{fig:taulepPNNscoreSR1_body-2} \gls{PNN} score distributions in the
			signal region of the \tauel sub-channel, for the chosen charged Higgs boson mass parameters.
			The lower panel of each plot shows the ratio of data to the \acrshort{SM} background prediction. The uncertainty bands include all statistical and systematic uncertainties.
			The normalization of the signal (shown for illustration) corresponds to the integral of the background.}
		\end{figure}

		\begin{figure}
			\centering
			\subfloat[\label{fig:taulepPNNscoreSR2_body_a}]{\includegraphics[width=0.45\linewidth]{chapters/chapter6_HPlus/images/taulep/clf_score_GB200_mass_80to80_SR_TAUMU.png}}
			\subfloat[\label{fig:taulepPNNscoreSR2_body_b}]{\includegraphics[width=0.45\linewidth]{chapters/chapter6_HPlus/images/taulep/clf_score_GB200_mass_130to130_SR_TAUMU.png}} \\
			\subfloat[\label{fig:taulepPNNscoreSR2_body_c}]{\includegraphics[width=0.45\linewidth]{chapters/chapter6_HPlus/images/taulep/clf_score_GB200_mass_160to160_SR_TAUMU.png}}
			\subfloat[\label{fig:taulepPNNscoreSR2_body_d}]{\includegraphics[width=0.45\linewidth]{chapters/chapter6_HPlus/images/taulep/clf_score_GB200_mass_200to200_SR_TAUMU.png}} \\
			% \subfloat[\label{fig:taulepPNNscoreSR2_body_e}]{\includegraphics[width=0.45\linewidth]{chapters/chapter6_HPlus/images/taulep/clf_score_GB200_mass_500to500_SR_TAUMU.png}}
			% \subfloat[\label{fig:taulepPNNscoreSR2_body_f}]{\includegraphics[width=0.45\linewidth]{chapters/chapter6_HPlus/images/taulep/clf_score_GB200_mass_3000to3000_SR_TAUMU.png}}
			\caption{\label{fig:taulepPNNscoreSR2_body-1} \gls{PNN} score distributions in the
			signal region of the \taumu sub-channel, for the chosen charged Higgs boson mass parameters.
			The lower panel of each plot shows the ratio of data to the \acrshort{SM} background prediction. The uncertainty bands include all statistical and systematic uncertainties.
			The normalization of the signal (shown for illustration) corresponds to the integral of the background.}
		\end{figure}

		\begin{figure}
			\centering
			% \subfloat[\label{fig:taulepPNNscoreSR2_body_a}]{\includegraphics[width=0.45\linewidth]{chapters/chapter6_HPlus/images/taulep/clf_score_GB200_mass_80to80_SR_TAUMU.png}}
			% \subfloat[\label{fig:taulepPNNscoreSR2_body_b}]{\includegraphics[width=0.45\linewidth]{chapters/chapter6_HPlus/images/taulep/clf_score_GB200_mass_130to130_SR_TAUMU.png}} \\
			% \subfloat[\label{fig:taulepPNNscoreSR2_body_c}]{\includegraphics[width=0.45\linewidth]{chapters/chapter6_HPlus/images/taulep/clf_score_GB200_mass_160to160_SR_TAUMU.png}}
			% \subfloat[\label{fig:taulepPNNscoreSR2_body_d}]{\includegraphics[width=0.45\linewidth]{chapters/chapter6_HPlus/images/taulep/clf_score_GB200_mass_200to200_SR_TAUMU.png}} \\
			\subfloat[\label{fig:taulepPNNscoreSR2_body_e}]{\includegraphics[width=0.45\linewidth]{chapters/chapter6_HPlus/images/taulep/clf_score_GB200_mass_500to500_SR_TAUMU.png}}
			\subfloat[\label{fig:taulepPNNscoreSR2_body_f}]{\includegraphics[width=0.45\linewidth]{chapters/chapter6_HPlus/images/taulep/clf_score_GB200_mass_3000to3000_SR_TAUMU.png}}
			\caption{\label{fig:taulepPNNscoreSR2_body-2} \gls{PNN} score distributions in the
			signal region of the \taumu sub-channel, for the chosen charged Higgs boson mass parameters.
			The lower panel of each plot shows the ratio of data to the \acrshort{SM} background prediction. The uncertainty bands include all statistical and systematic uncertainties.
			The normalization of the signal (shown for illustration) corresponds to the integral of the background.}
		\end{figure}

		\begin{figure}
			\centering
			% \subfloat[\label{fig:expected-limits-a}]{\includegraphics[width=.5\textwidth]{chapters/chapter6_HPlus/images/Limits/taujets_fullmass.eps}}
			% \subfloat[\label{fig:expected-limits-b}]{\includegraphics[width=.5\textwidth]{chapters/chapter6_HPlus/images/Limits/taulep_fullmass.eps}} \\
			% \subfloat[\label{fig:expected-limits-c}]{\includegraphics[width=\textwidth]{chapters/chapter6_HPlus/images/Limits/combined_fullmass.eps}}
			\subfloat[\label{fig:expected-limits-a}]{\includegraphics[width=.5\textwidth]{chapters/chapter6_HPlus/images/Limits/139_vs_36_analysis_limits_PNN_vs_BDT_TauEl_logx.png}} 
			\subfloat[\label{fig:expected-limits-b}]{\includegraphics[width=.5\textwidth]{chapters/chapter6_HPlus/images/Limits/139_vs_36_analysis_limits_PNN_vs_BDT_TauMu_logx.png}} \\
			\subfloat[\label{fig:expected-limits-c}]{\includegraphics[width=.75\textwidth]{chapters/chapter6_HPlus/images/Limits/139_vs_36_analysis_limits_PNN_vs_BDT_TauJets_logx.png}} 
			\caption{\label{fig:expected-limits} 
			Expected 95\% CL exclusion limits on $\sigma(\pp\to tb\Hpm)\times \mathrm{\cal{B}}(\Hpm \to \tau \nu)$ as a function of the charged Higgs boson mass in \LUMI of \pp collision data at \sqs in the \tauel signal region (a), the \taumu signal region (b), and the \taujets signal region (c). In the case of the expected limits, one- and two-standard-deviation uncertainty bands are also shown. As a comparison, the expected exclusion limits obtained with the dataset collected in 2015 and 2016~\cite{hpm-previous} are also shown.
			 }
		\end{figure}