This dissertation presents a search for resonant and non-resonant di-Higgs production in the $\gamma \gamma b \bar{b}$ final state using data from the ATLAS detector at the Large Hadron Collider (LHC).  The search is performed on data from proton-proton collisions at a center-of-mass energy of $\sqrt{s} = 13$ \TeV, using the full ATLAS Run 2 dataset. 

A long term goal of the LHC is to measure the Higgs self-coupling. 5$\sigma$ sensitivity to this value will only be possible with the 3 \invab expected at the end of the LHC, utilizing all di-Higgs decay channels and data from both ATLAS and CMS. Many extensions to the Standard Model predict enhancements to this value, which would alter the di-Higgs production rate. As such, limits are set on the self-coupling value in the non-resonant analysis. The resonant search targets a new, heavy scalar particle that decays into two Standard Model Higgs bosons.

The rate at which candidate events for this channel are selected can be improved with an increase in photon identification efficiency. Studies on photon identification and isolation are presented, with a focus on efficiency in future machine conditions. 

