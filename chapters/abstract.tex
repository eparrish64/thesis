This dissertation presents a search for resonant and non-resonant di-Higgs production in the $\gamma \gamma b \bar{b}$ final state using data from the ATLAS detector at the Large Hadron Collider (LHC).  The search is performed on 36.1 \ifb of data from proton-proton collisions at a center-of-mass energy of $\sqrt{s} = 13$ \TeV collected in 2015 and 2016.

No significant excesses are observed in this search. The non-resonant analysis sets limits on the $\hh\rightarrow\yybb$ \xsec times branching ratio, with an upper observed (expected) limit of 0.73 (0.93) pb. The observed (expected) limits on the Higgs boson trilinear coupling at 95\% \gls{CL} are set at $-8.2 < \klambda < 13.3$ ($-8.5 < \klambda < 13.7$). A model-independent resonant search is also presented, setting limits on a generic scalar resonance under the narrow-width approximation. These limits cover mass hypotheses ranging from 260 GeV to 1000 GeV, and the observed (expected) limits set are 0.85 (0.92) pb at the lowest mass hypothesis to 0.13 (0.15) pb at the highest mass hypothesis.

Work toward the future of this analysis is presented. Improvements in photon identification are studied, optimizing the photon identification menu through two approaches. First, through adding new discriminating variables to the menu through the addition of topological cluster moments, and second, through using a multivariate approach to define photon identification. Through these additional inputs and the employment of a Boosted Decision Tree (BDT), an improvement of as much as 27\% background rejection for the same signal efficiency as the current tight working point is shown.

Additionally, improvements to the analysis through the addition of a signal region targeting Vector Boson Fusion (VBF) production are shown. A multiclass BDT is used, with classes to model the VBF \hh production mode, along with gluon-gluon fusion \hh production, as well as the dominant \yy-continuum background, and $ttH$ mono-Higgs background. By adding this signal region, a 9.7\% improvement in Asimov significance is achieved using the full 140 \ifb of \RunTwo data. 