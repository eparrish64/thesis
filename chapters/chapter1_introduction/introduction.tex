\chapter{Introduction}\label{chap:Intro}
	Science has long sought to provide an explanation to the most fundamental questions. The field of particle physics tries to answer  the question of ``What are we made of?'' The \gls{SM} of particle physics is an attempt at combining all of the fundamental forces and particles into one mathematically based theory. The \gls{SM} has been rigorously tested and for the most part, holds up except for a few key issues. To name a few of these issues: the absence of gravity in the theory, no candidate for dark matter, neutrino masses, and the matter-antimatter asymmetry of the universe.

	There are many \gls{BSM} theories that address these issues and more. A common theme of these models is an extended Higgs sector, meaning more Higgs boson-like particles. This dissertation focuses on a search for an additional Higgs boson that carries electromagnetic charge, \Hpm. The search is performed on data collected with the \acrshort{ATLAS} detector on the \gls{LHC} at \gls{CERN} in Geneva, Switzerland. 

	A theoretical motivation is given in Chapter \ref{chap:Theory} followed by a detailed explanation of the \gls{LHC} and the \gls{ATLAS} detector in Chapter \ref{chap:experiment}. Simulation of particle collisions and the reconstruction of both simulated and actual data is detailed in Chapters \ref{chap:sim} and \ref{chap:reco} respectively. Chapter \ref{chap:hpana} provides a thorough description of the search for the \Hpm in the \taunu final state. Lastly future works are discussed in Chapter \ref{chap:conclusions}.

	A considerable amount of the author's time in the Ph.D. program was dedicated towards the search described in this dissertation and acting as the Data Quality Co-Coordinator for the hadronic calorimeter within the \gls{ATLAS} Collaboration. In addition, the author assisted on various projects within the hadronic calorimeter group including two test beams and maintenance of the calorimeter within the larger ATLAS detector.