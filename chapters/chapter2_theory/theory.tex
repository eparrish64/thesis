\chapter{Theory}\label{chap:Theory}
    In this chapter, the theoretical motivation of a search for \HpmLong is described. A review of the \acrfull{SM} of particle physics is laid out, followed by a brief overview of \gls{SUSY} focusing on the \acrfull{MSSM}. Finally, the Type II \acrfull{2HDM} relation to the \Hpm production cross section and subsequent branching ratio into \gls{SM} particles is described as motivation for the choice of studying \HpmLong.

\section{The Standard Model}\label{sec:SM}
	The \gls{SM} of particle physics is a \gls{QFT} that describes all known matter and forces\footnote{Notably, the SM does not describe gravity.}. The \gls{SM} describes forces as exchanges of particles called bosons and is described in detail in Section \ref{ssec:Interactions}. These interactions occur between fundamental particles called fermions that comprise the known matter of the universe. Figure \ref{fig:atom-to-quarks} shows how fermions called quarks are the building blocks of nucleons, and thus atoms.

	\begin{figure}[!ht]
		\centering
		\includegraphics[width=.75\textwidth,keepaspectratio=true]{chapters/chapter2_theory/images/Atom_to_Quark_Cartoon.png}
		\caption{Quarks are fundamental particles that combine to create hadrons (protons, neutrons, $\pi^{0,\pm}$, etc.) \cite{atom-to-quark}.}
		\label{fig:atom-to-quarks}
	\end{figure}

	\subsection{Particles}\label{ssec:Particles}
		The particles that make up the \gls{SM} are defined by their properties, or quantum numbers. These quantum numbers are used to categorize particles into various types. Intrinsic angular momentum charge, or spin, is the first quantum number that separates particles into fermions or bosons. Fermions are those that carry half-integer spin, and thus obey Fermi-Dirac statistics, while bosons carry full integer spin values and obey Bose-Einstein statistics.
		
		\subsubsection{Fermions}\label{sssec:Fermions}
			The matter encountered in everyday life is comprised of fermions. Fermions are subdivided into two groups, quarks and leptons. The quarks participate in the strong interaction via their color charge. Quarks cannot exist as an isolated particle and thus combine into hadrons in a process called hadronization; the bound states they form are colorless. The proton and neutron are examples of hadrons. Hadrons, quarks, and their interaction with the strong force are detailed in Section \ref{sssec:QCD}. Leptons carry no color charge and therefore do not participate in strong force interactions. The fermions in the \gls{SM} all participate in the electroweak interaction. However, the electromagnetic interaction is limited to those fermions that carry an electromagnetic charge. Section \ref{sssec:Electroweak} describes the electroweak interaction in detail.

			Fermions can then be further divided into three generations, each lepton has an electrically neutral weak force partner in the form of a neutrino. Table \ref{tab:fermions} lists all the \gls{SM} fermions and their properties. Every particle has a partner with identical properties except for an opposite \gls{EM} charge. These partners are called antimatter and are denoted with a bar above the particle symbol $(u, \bar{u})$.

			\begin{table}[!thp]
				% \centering
				\caption{Standard Model fermions and their properties \cite{pdg}.}
				\resizebox{\textwidth}{!}{\begin{tabular}{c | c | c | c | c | c | c | c |}
				\cline{2-8}

																& \begin{tabular}[c]{@{}c@{}}$1^{st}$ \\ Generation \end{tabular} & \begin{tabular}[c]{@{}c@{}} $2^{nd}$ \\ Generation \end{tabular} 	& \begin{tabular}[c]{@{}c@{}} $3^{rd}$ \\ Generation \end{tabular}	& Spin 			& \begin{tabular}[c]{@{}c@{}}EM \\Charge \end{tabular}		& Color 		& Mass \\[1ex] \hline
				\multicolumn{1}{|c|}{\multirow{3}{*}{Quarks}}   & Up (u)						& Charm (c)						& Top (t)						& $\frac{1}{2}$ & $+\frac{2}{3}$	& \scalecheck  	& \begin{tabular}[c]{@{}c@{}} $m_u = 2.16^{+0.49}_{-0.26}$ MeV \\[1ex] $m_c = 1.27 \pm 0.02$ GeV \\[1ex] $m_t = 172.76 \pm 0.30$ GeV \end{tabular}		\\[1ex] \cline{2-8}
				\multicolumn{1}{|l|}{}                         	& Down (d)						& Strange (s)					& Bottom (b) 					& $\frac{1}{2}$ & $-\frac{1}{3}$	& \scalecheck	& \begin{tabular}[c]{@{}c@{}} $m_d = 4.67^{+0.48}_{-0.17}$ MeV \\[1ex] $m_s = 93^{11}_{-5}$ MeV \\[1ex] $m_b = 4.18^{0.03}_{-0.02}$ GeV \end{tabular}				\\[1ex]	\hline
				\multicolumn{1}{|c|}{\multirow{3}{*}{Leptons}}  & Electron ($e^{-}$)			& Muon ($\mu^{-}$)				& Tau ($\tau^{-}$)				& $\frac{1}{2}$ & $-1$ 				& X 			& \begin{tabular}[c]{@{}c@{}} $m_{e^{-}} = 0.51$ MeV \\[1ex] $m_{\mu^{-}} = 105.65$ MeV \\[1ex] $m_{\tau^{-}} = 1776.86 \pm 0.12$ MeV \end{tabular}		\\[1ex] \cline{2-8}
				\multicolumn{1}{|c|}{}  						& \begin{tabular}[c]{@{}c@{}}Electron \\ Neutrino\end{tabular} ($\nu_{e}$)	& \begin{tabular}[c]{@{}c@{}}Muon \\ Neutrino\end{tabular} ($\nu_{\mu}$)	& \begin{tabular}[c]{@{}c@{}}Tau \\ Neutrino\end{tabular} ($\nu_{\tau}$) & $\frac{1}{2}$ & $0$ 				& X 			& \begin{tabular}[c]{@{}c@{}} $m_{\nu_{e}} < 1.1$ eV \\[1ex] $m_{\nu_{\mu}} < 0.19 $ MeV  \\[1ex] $m_{\nu_{\tau}} < 18.2 $ MeV \end{tabular}		\\[1ex] \hline			
				\end{tabular}}
				\label{tab:fermions}
			\end{table}


		\subsubsection{Bosons}\label{sssec:Bosons}
			Bosons are colloquially referred to as force-carriers in that the fundamental forces act via exchanging gauge bosons. This means that each force has associated boson(s) which is described by a field theory. The electroweak \gls{QFT} is more complicated, and is described in detail in section \ref{sssec:Electroweak}. Table \ref{tab:bosons} lists the \gls{SM} bosons\footnote{Excluding the Higgs boson.}, their associated field theory and properties.

			\begin{table}[!thp]
			\centering
			\caption{Standard Model bosons and their properties \cite{pdg}.}
			\resizebox{\textwidth}{!}{\begin{tabular}{| c | c | c | c | c | c |}  
			\hline
			\multicolumn{1}{|c|}{Field Theory}							& Boson 				& Spin 	& \begin{tabular}[c]{@{}c@{}} EM \\ Charge \end{tabular} 	& Color 		& Mass 	\\[1ex] \hline 
			\multicolumn{1}{|c|}{\acrfull{QCD}}			& Gluon (g)				& 1 	& 0 														& \scalecheck 	& 0		\\[1ex] \hline
 			% \multicolumn{1}{|c|}{\acrfull{QED}} 		& Photon ($\gamma$) 	& 1 	& 0 													 	& X 			& $< 1 \, \mathrm{x} \, 10^{-18}$ eV  	\\[1ex] \hline
 			\multicolumn{1}{|c|}{\acrfull{QED}} 		& Photon ($\gamma$) 	& 1 	& 0 													 	& X 			& 0  	\\[1ex] \hline
			\multicolumn{1}{|c|}{\multirow{2}{*}{Electroweak Theory}} 	& $W^{\pm}$ 			& 1 	& $\pm 1$													& X 			& $80.379 \pm 0.012$ GeV	\\[1ex] \cline{2-6}
			\multicolumn{1}{|c|}{} 										& $Z^{0}$				& 1 	& 0 													 	& X 			& $91.1876 \pm 0.0021$ GeV  	\\[1ex] \hline
			\end{tabular}}
			\label{tab:bosons}
			\end{table}

	\subsection{Interactions}\label{ssec:Interactions}
		 The \gls{SM} is based upon conservation laws. These conversation laws are what dictate the allowed interactions of matter. Lepton generation number\footnote{Ignoring neutrino oscillations.}, electric charge, color charge, 4-momentum ($p=(E,\vec{p})$), and angular momentum are all conserved in the \gls{SM}. In strong interactions baryon number\footnote{Here, $n_{q}$ and $n_{\bar{q}}$ are the number of quarks and antiquarks that comprise the baryon.} $(B = \frac{1}{3}(n_{q} - n_{\bar{q}}) )$ is also conserved.
		% The \gls{SM} is built upon a gauge group of type $SU(3)_C \times SU(2)_L \times U(1)_Y$. The $SU(3)_C$ term dictates the strong interaction while the $SU(2)_L \times U(1)_Y$ term describes the electroweak interaction.
		 % At its core, the \gls{SM} relies upon symmetries; from these symmetries, conservation laws follow. It is these laws of conservation that dictate the allowed interactions of matter. The symmetry between charge conjugation and mirror reflection \gls{CP} can be broken in certain circumstances, but holds in strong and electromagnetic interactions. The breaking of \gls{CP} symmetry occurs in the weak interaction and implies an asymmetry between matter and antimatter. Since this symmetry holds for strong and electromagnetic interactions, baryon number\footnote{Here, $n_{q}$ and $n_{\bar{q}}$ are the number of quarks and antiquarks that comprise the baryon.} $(B = \frac{1}{3}(n_{q} - n_{\bar{q}}) )$ and lepton number are conserved in \gls{SM} interactions. Lepton generation number\footnote{Ignoring neutrino oscillations}, electric charge, color charge, 4-momentum ($p=(E,\vec{p})$), and angular momentum are all conserved in the \gls{SM}.
		 % The first, being a symmetry under charge conjugation, mirror reflection, and time reversal is known as \gls{CPT} symmetry.

		\subsubsection{Quantum Electrodynamics}\label{sssec:QED}
			The electromagnetic force is governed by the \gls{QFT} known as \acrfull{QED}. This force is mediated by the photon, $\gamma$, a massless boson with \gls{EM} charge 0. The \gls{EM} force only interacts with electrically charged particles, including all quarks and the $e$, $\mu$, and $\tau$ leptons.

		\subsubsection{Electroweak Interaction}\label{sssec:Electroweak}
			The weak force is most often seen in nuclear decays and is mediated by the $W^{\pm}$ and $Z^0$ bosons. Due to the relatively large mass of these bosons, the weak force has a very limited range. The weak force interacts via the quantum number called weak isospin ($T$). The $W^{\pm}$ affects the third component of weak isospin ($T_3$), thus only coupling to so-called left-handed fermions. In this way, $T_{3}$ defines the ``handedness'', or chirality of a particle. At energies $> 100 $ GeV the electromagnetic and weak forces combine into the electroweak force. Isospin and another quantum number hypercharge combine to give \gls{EM} charge. $Q_{EM} = T_3 + \frac{1}{2} Y_W$. Table \ref{tab:weak} contains the allowed values for weak isospin and hypercharge ($Y_W$). 

			\begin{table}[!thp]
					\centering
					\caption{Standard Model fermions and their Electroweak properties \cite{pdg}.}
					\resizebox{\textwidth}{!}{\begin{tabular}{| c | c | c | c | c | c | c | c | c | c | c | c |}
					\hline

																	& \begin{tabular}[c]{@{}c@{}}$1^{st}$ \\ Generation \end{tabular} & \begin{tabular}[c]{@{}c@{}} $2^{nd}$ \\ Generation \end{tabular} 	& \begin{tabular}[c]{@{}c@{}} $3^{rd}$ \\ Generation \end{tabular}		& \begin{tabular}[c]{@{}c@{}}EM \\ Charge \end{tabular} & \multicolumn{2}{|c|}{$Y_{W}$} & \multicolumn{2}{|c|}{T} 	& \multicolumn{2}{|c|}{$T_{3}$} \\ \hline
					& & & & &  LH 				& RH 					& LH 			& RH 				& LH 	& RH \\ \cline{6-11}
					\multicolumn{1}{|c|}{\multirow{3}{*}{Quarks}}   & Up (u)						& Charm (c)						& Top (t)						&  $+\frac{2}{3}$ & $+\frac{1}{3}$	& $+\frac{4}{3}$		& $\frac{1}{2}$	& 0					& $\pm \frac{1}{2}$	 	& 0	 \\[1ex] \cline{2-11}
					\multicolumn{1}{|l|}{}                         	& Down (d)						& Strange (s)					& Bottom (b) 					&  $-\frac{1}{3}$ & $+\frac{1}{3}$	& $-\frac{2}{3}$		& $\frac{1}{2}$	& 0					& $\pm \frac{1}{2}$	 	& 0	 \\[1ex] \hline
					\multicolumn{1}{|c|}{\multirow{3}{*}{Leptons}}  & Electron ($e^{-}$)			& Muon ($\mu^{-}$)				& Tau ($\tau^{-}$)				&  $-1$ & $-1$				& $0$					& $\frac{1}{2}$	& 0					& $\pm \frac{1}{2}$	 	& 0	 \\[1ex] \cline{2-11}
					\multicolumn{1}{|c|}{}  						& \begin{tabular}[c]{@{}c@{}}Electron \\ Neutrino\end{tabular} ($\nu_{e}$)	& \begin{tabular}[c]{@{}c@{}}Muon \\ Neutrino\end{tabular} ($\nu_{\mu}$)	& \begin{tabular}[c]{@{}c@{}}Tau \\ Neutrino\end{tabular} ($\nu_{\tau}$) & $0$ & $-1$				& $-2$					& $\frac{1}{2}$	& 0					& $\pm \frac{1}{2}$	 	& 0	 \\ \hline			
					\end{tabular}}
					\label{tab:weak}
				\end{table}

			The $W^\pm$ bosons have a $T_3$ component of weak isospin and act as raising or lowering operators on the $T_3$ component of left handed fermions. The $Z$ boson does not have a $T_3$ component, and thus does not act on weak isospin of fermions. The Z boson instead transfers momentum, energy, and spin on all fermions irregardless of their chirality. 

			% \begin{table}[!thp]
			% 	\centering
			% 	\caption{Standard Model particles and their electroweak quantum numbers \cite{pdg}}
			% 	\begin{tabular}{| c | c | c | c | c | c | c |}  
			% 	\hline
			% 	Particle 	& \multicolumn{2}{|c|}{$Y_{W}$} & \multicolumn{2}{|c|}{T} 	& \multicolumn{2}{|c|}{$T_{3}$} \\ \hline
			% 				& LH 				& RH 					& LH 			& RH 				& LH 	& RH \\ \hline
			% 	u 			& $+\frac{1}{3}$	& $+\frac{4}{3}$		& $\frac{1}{2}$	& 0					& $\pm \frac{1}{2}$	 	& 0	 \\[1ex] \hline
			% 	d 			& $+\frac{1}{3}$	& $-\frac{2}{3}$		& $\frac{1}{2}$	& 0					& $\pm \frac{1}{2}$	 	& 0	 \\[1ex] \hline
			% 	c 			& $+\frac{1}{3}$	& $+\frac{4}{3}$		& $\frac{1}{2}$	& 0					& $\pm \frac{1}{2}$	 	& 0	 \\[1ex] \hline
			% 	s 			& $+\frac{1}{3}$	& $-\frac{2}{3}$		& $\frac{1}{2}$	& 0					& $\pm \frac{1}{2}$	 	& 0	 \\[1ex] \hline
			% 	t 			& $+\frac{1}{3}$	& $+\frac{4}{3}$		& $\frac{1}{2}$	& 0					& $\pm \frac{1}{2}$	 	& 0	 \\[1ex] \hline
			% 	b 			& $+\frac{1}{3}$	& $-\frac{2}{3}$		& $\frac{1}{2}$	& 0					& $\pm \frac{1}{2}$	 	& 0	 \\[1ex] \hline
			% 	e 			& $-1$				& $0$					& $\frac{1}{2}$	& 0					& $\pm \frac{1}{2}$	 	& 0	 \\[1ex] \hline
			% 	$\nu_e$ 	& $-1$				& $-2$					& $\frac{1}{2}$	& 0					& $\pm \frac{1}{2}$	 	& 0	 \\[1ex] \hline
			% 	$\mu$ 		& $-1$				& $0$					& $\frac{1}{2}$	& 0					& $\pm \frac{1}{2}$	 	& 0	 \\[1ex] \hline
			% 	$\nu_\mu$ 	& $-1$				& $-2$					& $\frac{1}{2}$	& 0					& $\pm \frac{1}{2}$	 	& 0	 \\[1ex] \hline
			% 	$\tau$ 		& $-1$				& $0$					& $\frac{1}{2}$	& 0					& $\pm \frac{1}{2}$	 	& 0	 \\[1ex] \hline
			% 	$\nu_\tau$ 	& $-1$				& $-2$					& $\frac{1}{2}$	& 0					& $\pm \frac{1}{2}$	 	& 0	 \\[1ex] \hline
			% 	$\gamma$ 	& \multicolumn{2}{|c|}{$0$}					& $\frac{1}{2}$	& 0					& $\pm \frac{1}{2}$	 	& 0	 \\[1ex] \hline
			% 	$g$ 		& \multicolumn{2}{|c|}{X}					& $\frac{1}{2}$	& 0					& $\pm \frac{1}{2}$	 	& 0	 \\[1ex] \hline
			% 	$W$ 		& \multicolumn{2}{|c|}{$0$}					& $\frac{1}{2}$	& 0					& $\pm \frac{1}{2}$	 	& 0	 \\[1ex] \hline
			% 	$Z$ 		& \multicolumn{2}{|c|}{$0$}					& $\frac{1}{2}$	& 0					& $\pm \frac{1}{2}$	 	& 0	 \\[1ex] \hline
			% 	$H$ 		& \multicolumn{2}{|c|}{$+1$}				& $\frac{1}{2}$	& 0					& $\pm \frac{1}{2}$	 	& 0	 \\[1ex] \hline
			% 	\end{tabular}
			% 	\label{tab:weak}
			% 	\end{table}

		\subsubsection{Quantum Chromodynamics}\label{sssec:QCD}
		
			\acrfull{QCD} is the \gls{QFT} that describes the strong force which holds together atomic nuclei and other objects called hadrons. The strong force interacts via the color charge\footnote{This color does is not the visual color we are used to; merely a convenient analogous naming scheme.} which can have values of either red, green, or blue. Particles that have a color charge cannot exist on their own, they must form colorless bound states called hadrons. Since the strong force grows with distance, if a quark is ejected out from a hadron, the stored energy is such that new particles with color charge will be spontaneously created from the vacuum, binding with the free quark in a process called hadronization. In a particle detector, the hadronization process cascades and creates showers of hadrons that are reconstructed as so called jets.

	\subsection{The Higgs Mechanism}\label{ssec:Higgs}

		The Higgs field is the mass generator of the \gls{SM} and was first theorized by Peter Higgs \cite{Higgs-paper}, François Englert, and Robert Brout \cite{Englert-Brout} in 1964.  The \gls{SM} itself has four massless bosons, $B$ and $\vec{W}$ $(W_{1,2,3})$, that do not correspond to the observed bosons. Instead, the Higgs mechanism couples to them via a complex scalar doublet ($\phi$): 
		\begin{equation}\label{eqn:scal doub} \phi = \binom{\phi^+}{\phi^0}\end{equation}
		The scalar potential that gives rise to this phenomena can be written as 
		\begin{equation}\label{eqn:higgs potential} V(\phi) = \mu^2 |\phi^{\dagger}\phi| + \lambda (|\phi^{\dagger}\phi|)^2\end{equation}
		When $\mu^2>0$ and $\lambda>0$ the minimum of the potential $V(\phi)$ is 0. 
		\begin{figure}[!ht] \centering \includegraphics[width=.7\textwidth,keepaspectratio=true]{chapters/chapter2_theory/images/higgspotential.png} \caption{The Higgs potential defined in Equation \ref{eqn:higgs potential} with $\mu^2<0$ \cite{Higgs-phys}.} \label{fig:higgs-potential}\end{figure}
		However, when $\mu^2<0$, the scalar potential $V(\phi)$ takes the shape shown in Figure \ref{fig:higgs-potential}.
		It follows that the \gls{VEV} of $\phi$ is then 
		\begin{equation}\label{eqn:higgs vev} \langle \phi \rangle = \sqrt{\frac{-\mu^2}{2\lambda}} = \frac{\nu}{\sqrt{2}}	\end{equation}
		where $\nu = \sqrt{\frac{-\mu^2}{\lambda}}$.
		From here, convention states to choose an arbitrary direction of the fluctuation as 
		\begin{equation}\label{eqn:phi zero} \phi^0 = \frac{1}{\sqrt{2}} \binom{0}{\nu} \end{equation}
		By choosing these values three of the bosons are absorbed in giving mass to the $W^{\pm}$ and $Z^0$ bosons leaving the final as the real scalar field $h(x)$
		\begin{equation}\label{eqn:phi-h} \phi(x) = \phi^0 + h(x) \end{equation}
		Substituting the definition of $\phi^0$ yields
		\begin{equation}\label{eqn:phi-h-vec} \phi = \frac{1}{\sqrt{2}} \binom{0}{\nu+h(x)} \end{equation}
		which couples to the \gls{SM} bosons via
		\begin{equation}\label{eqn:coupling} \left(\frac{1}{2} g \vec{\sigma} \cdot \vec{W} + \frac{1}{2} g^\prime B \right) \phi^0  \end{equation} where $\vec{\sigma}$ are the Pauli matrices, $g$ is the weak coupling constant, and $g^{\prime}$ is the hypercharge coupling constant. From this coupling, there are four eigenstates which correspond to the observed bosons
		\begin{equation}\label{eqn:mass-eigenstates} \begin{split}
		W^\pm = \frac{1}{\sqrt{2}} ( W^1_\mu \mp i W^2_\mu ) \\
		Z^\mu = \frac{ - g^\prime B_\mu + g W^3_\mu }{ \sqrt{g^2+g^{\prime 2}} } \\
		A^\mu = \frac{ g B_\mu + g^\prime W^3_\mu }{ \sqrt{g^2+g^{\prime 2}} }
		\end{split}
		\end{equation}
		These eigenstates have corresponding mass values of 
		\begin{equation}\label{eqn:mass-eigenstates-masses} \begin{split}
		M^2_W = \frac{1}{4}g^2\nu^2 \\
		M^2_Z = \frac{1}{4}(g^2+g\prime^2)\nu \\
		M^2_A = 0
		\end{split}
		\end{equation}
		The eigenstate labeled here as $A$ is the photon. The Higgs boson was discovered in 2012 by the \gls{ATLAS} and \gls{CMS} collaborations at \gls{CERN} with a mass of $125$ GeV \cites{higgs-discovery-atlas}{CMS-Higgs-Discovery}. The \gls{ATLAS} result in the $H \to \gamma \gamma$ can be seen in Figure \ref{fig:higgs-discovery}. The scalar boson that was found appears to be the \gls{SM} Higgs Boson with the properties shown in Table \ref{tab:higgs-properties}.

		\begin{table}[!thp]
			\centering
			\caption{The Higgs boson's properties \cite{pdg}.}
			\resizebox{.8\textwidth}{!}{\begin{tabular}{| c | c | c | c | c | c | c | c |}  
			\hline
			\multicolumn{1}{|c|}{Field Theory}				& Boson 				& Spin 	& \begin{tabular}[c]{@{}c@{}} EM \\ Charge \end{tabular} 	& Color 		& Mass 	 					& $Y_{W}$ 			& $T_{3}$ 	\\ \hline 
			\multicolumn{1}{|c|}{Higgs Mechanism}			& Higgs (H)				& 0 	& 0 														& X 			& $125.25 \pm 0.17$ GeV		& $\pm\frac{1}{2}$	& $\mp 1$	\\[1.5ex] \hline
			\end{tabular}}
			\label{tab:higgs-properties}
		\end{table}

		% \begin{figure}[!ht]
		% \centering
		% \includegraphics[width=.7\textwidth,keepaspectratio=true]{chapters/chapter2_theory/images/Higgs_Discovery_gam_gam.jpeg}
		% \caption{The distributions of the invariant mass of diphoton candidates after all selections for the combined 7 TeV and 8 TeV data sample. The inclusive sample is shown in (a) and a weighted version of the same sample in (c); the weights are explained in the text. The result of a fit to the data of the sum of a signal component fixed to $m_H=126.5$ GeV  and a background component described by a fourth-order Bernstein polynomial is superimposed. The residuals of the data and weighted data with respect to the respective fitted background component are displayed in (b) and (d). \cite{higgs-discovery-atlas}}
		% \label{fig:higgs-discovery}
		% \end{figure}
		\begin{figure}[!ht]
		\centering
		\includegraphics[width=.7\textwidth,keepaspectratio=true]{chapters/chapter2_theory/images/Higgs_Discovery_gam_gam.jpeg}
		\caption{The distributions of the invariant mass of diphoton candidates after all selections for the combined 7 TeV and 8 TeV data sample. The result of a fit to the data of the sum of a signal component fixed to $m_H=126.5$ GeV  and a background component described by a fourth-order Bernstein polynomial is superimposed. Taken from \cite{higgs-discovery-atlas}.}
		\label{fig:higgs-discovery}
		\end{figure}

\section{Supersymmetry}\label{sec:SUSY}
	While the \gls{SM} describes a wide range of physics to a high degree of accuracy, it is not without issues. For instance, the \gls{SM} does not offer an explanation for gravity, dark matter, or the observed matter-antimatter asymmetry of the universe. In addition, the \gls{SM} predicts the mass of neutrinos to be 0. Observed neutrino mixing, where $\nu_e \to \nu_\mu$, $\nu_\tau \to \nu_\mu$, etc., contradicts this; neutrinos must have mass \cite{pdg}.

	One promising model that offers solutions to many of these issues is \gls{SUSY}. As discussed previously, the \gls{SM} is built upon symmetries, and the breaking of these symmetries gives us electroweak unification. \gls{SUSY} proposes another symmetry, this time between fermions and bosons. 
	\begin{equation}\label{eqn:SUSY}
	\begin{split}
		Q | Fermion \rangle = | Boson \rangle, \\
		Q | Boson \rangle = | Fermion \rangle
	\end{split}
	\end{equation}
	Equation \ref{eqn:SUSY} shows how the \gls{SUSY} operator Q acts on particles. Here, Q provides a bosonic supersymmetric partner to every fermion and vice versa. \gls{SUSY} naturally offers solutions to the ``hierarchy problem'' with the \gls{SM}. 

	The hierarchy problem arises from the difference in electroweak ($M_W\sim100$ GeV)  and Planck ($M_P\sim2.4\, \mathrm{x} \, 10^{18}$ GeV) mass scales. For the Higgs mass to be on the scale of $M_H \sim 125$ GeV incredibly large and small mass terms must cancel perfectly, leading to a feeling of ``unnaturalness''. \gls{SUSY} brings many new particles into the picture, theorized to occupy the intermediate mass range leading to a more natural theory.

	\subsection{\acrlong{MSSM} Particles}\label{ssec:MSMM}
		\gls{SUSY} is a large group of theories that include theories with various numbers of additional superpartner particles. The \gls{MSSM} is the smallest extension of the \gls{SM} that introduces \gls{SUSY}. In the \gls{MSSM}\footnote{As well as all other \gls{SUSY} models.}, each \gls{SM} particle is part of a supermultiplet with its superpartner where both particles have the same quantum numbers, except spin. If this supersymmetry is unbroken, then the superpartner and the \gls{SM} particle would have the same mass as well. However, \gls{SUSY} has not been observed, so the supersymmetry must be broken putting the mass scale on the TeV scale. Table \ref{tab:MSSM} lists the \gls{MSSM} supermultiplets and the associated naming conventions.

			% \begin{table}[!thp]
			% 	\centering
			% 	\caption{\gls{SM} particles and their \gls{MSSM} partners \cite{pdg}.}
			% 	\begin{tabular}{| l | c | c |}
			% 	\hline
			% 	Name 				& \gls{SM} 	& \gls{MSSM} \\[1ex] \hline
			% 	\multicolumn{3}{|c|}{Spin-$\frac{1}{2}$ quarks and spin-$0$ squarks} \\[1ex] \hline
			% 	(s)up 				& $u$ 	& $\tilde{u}$ \\[1ex] \hline
			% 	(s)down 			& $d$ 	& $\tilde{d}$ \\[1ex] \hline
			% 	(s)charm 			& $c$ 	& $\tilde{c}$ \\[1ex] \hline
			% 	(s)strange 			& $s$		& $\tilde{s}$ \\[1ex] \hline
			% 	(s)top 				& $t$ 	& $\tilde{t}$ \\[1ex] \hline
			% 	(s)bottom 			& $b$ 	& $\tilde{b}$ \\[1ex] \hline
			% 	\multicolumn{3}{|c|}{Spin-$\frac{1}{2}$ leptons and spin-$0$ sleptons} \\[1ex] \hline
			% 	(s)electron 		& $e$ 	& $\tilde{e}$ \\[1ex] \hline
			% 	(s)electron (s)neutrino 	& $\nu_e$ 	& $\widetilde{\nu_e}$ \\[1ex] \hline
			% 	(s)muon 			& $\mu$ 	& $\tilde{\mu}$ \\[1ex] \hline
			% 	(s)muon (s)neutrino & $\nu_\mu$ 	& $\widetilde{\nu_\mu}$ \\[1ex] \hline
			% 	(s)tau 				& $\tau$ 	& $\tilde{\tau}$ \\[1ex] \hline
			% 	(s)tau (s)neutrino 	& $\nu_\tau$ 	& $\widetilde{\nu_\tau}$ \\[1ex] \hline
			% 	\multicolumn{3}{|c|}{Spin-$0$ Higgs and spin-$\frac{1}{2}$ Higgsinos} \\[1ex] \hline
			% 	Higgs(ino)			& $H$ 	& $\tilde{H}$ \\[1ex] \hline
			% 	gluon (gluino) 		& $g$ 	& $\tilde{g}$ \\[1ex] \hline
			% 	W (Wino) 			& $W^{\pm}$, $W^0$ & $\widetilde{W^\pm}, \widetilde{W^0}$ \\[1ex] \hline
			% 	B (Bino) 			& $B^0$ & $\widetilde{B^0}$ \\[1ex] \hline

 		% 		\end{tabular}
			% 	\label{tab:MSSM}
			% \end{table}

			\begin{table}[!thp]
				\centering
				\caption{\gls{SM} particles and their \gls{MSSM} partners \cite{pdg}.}
				\begin{tabular}{| c | c |}
				\hline
				\gls{SM} 	& \gls{MSSM} \\[1ex] \hline
				\multicolumn{2}{|c|}{Spin-$\frac{1}{2}$ quarks and spin-$0$ squarks ($\times 3$ generations)} \\[1ex] \hline
				$(u_{L} \, d_{L})$ 					& $(\tilde{u}_{L} \, \tilde{d}_{L})$ \\[1ex]
				$u^{\dagger}_{R}$ 					& $\bar{u}^{*}_{R}$ \\[1ex]
				$d^{\dagger}_{R}$ 					& $\bar{d}^{*}_{R}$ \\[1ex] 
				\hline \hline

				\multicolumn{2}{|c|}{Spin-$\frac{1}{2}$ leptons and spin-$0$ sleptons ($\times 3$ generations)} \\[1ex] \hline
				${\nu_{L} \, e_{L}}$ 				& $(\tilde{\nu}_{L} \,  \tilde{e}_{L}$ \\[1ex]
				$e^{\dagger}_{R}$ 					& $\bar{e}^{*}_{R}$ \\[1ex]

				\hline \hline
				\multicolumn{2}{|c|}{Spin-$0$ Higgs and spin-$\frac{1}{2}$ Higgsinos} \\[1ex] \hline
				$(H^{\dagger}_{u} 	\, H^{0}_{u})$ 	& $(\tilde{H}^{+}_{u} \, \tilde{H}^{0}_{u} )$ \\[1ex]
				$(H^{0}_{d} 		\, H^{-}_{d})$ 	& $(\tilde{H}^{0}_{d} \, \tilde{H}^{-}_{d} )$ \\[1ex]

				\hline \hline
				\multicolumn{2}{|c|}{Spin-$1$ gauge bosons and spin-$\frac{1}{2}$ gauginos} \\[1ex] \hline
				$g$									& $\tilde{g}$ \\[1ex]
				$(W^{\pm} \, W^{0})$ 				& $(\tilde{W}^{\pm} \, \tilde{W}^{0})$ \\[1ex]
				$B^{0}$ 							& $\tilde{B}^{0}$ \\[1ex]
				\hline \hline

 				\end{tabular}
				\label{tab:MSSM}
			\end{table}


	\subsection{2 Higgs Doublet Model}\label{ssec:2HDM}
		Having only one Higgs chiral supermultiplet with hypercharge $Y_W=\pm \frac{1}{2}$ leads to a gauge anomaly \cite{2HDM}. This can be resolved by introducing two Higgs doublets with hypercharge $Y_W=\frac{1}{2}$ and $Y_W=-\frac{1}{2}$. Such is the case in the \gls{MSSM} which requires two complex doublet scalar fields where one couples to the up-type quarks and the other couples to down-type quarks and charged leptons. The \gls{MSSM} Higgs sector has 8 degrees of freedom. Following the same type of mechanism described in Section \ref{ssec:Higgs} three of these degrees of freedom give the observed $W^\pm$ and $Z^0$ bosons. 
		\begin{table}[!thp]
				\centering
				\caption{\gls{2HDM} extended Higgs sector \cite{2HDM}}
				\begin{tabular}{| l | c |}
				\hline
				light neutral scalar 	& $h^0$ \\ \hline
				heavy neutral scalar 	& $H^0$ \\ \hline
				neutral pseudoscalar 	& $A^0$ \\ \hline
				two charged scalars 	& \Hpm \\ \hline
 				\end{tabular}
				\label{tab:2HDM}
		\end{table}
		This leaves the extended Higgs sector shown in Table \ref{tab:2HDM}, where the $h^0$ is a SM-like Higgs. The boson discovered by the \gls{ATLAS} and \gls{CMS} collaborations in 2012 is consistent with the $h^{0}$. When referring to the charged Higgs bosons, we often refer to them using one symbol \Hpm. In the \gls{2HDM} there are two free parameters\footnote{There are more free parameters with regards to the full \gls{2HDM}. These are the two regarding the charged Higgs bosons that are relevant for the rest of this dissertation.}, the masses of the \Hpm and the ratio of their vacuum expectation values which is defined as \tanb. At the time of writing, the extended Higgs sector is an active area of research with many new searches actively being performed \cite{pdg}. The most recent results from ATLAS can be seen in Reference \cite{ATLAS-HBSM-Summary}.

		% These types of models are referred to as Type II \gls{2HDM} \cite{2HDM}.


\section{Charged Higgs Bosons}\label{sec:Hpm}
	Since the \Hpm couplings are proportional to the fermion masses, the main production modes at the LHC are through \ttbar and $Wt$ diagrams where the $W^{\pm}$ boson is replaced by a \Hpm.
	\begin{figure}[!ht]
		\centering
		\subfloat[\label{fig:hpm-diagrams_a}]{\includegraphics[width=0.3\textwidth]{chapters/chapter2_theory/images/NonResonant.pdf}}
		\subfloat[\label{fig:hpm-diagrams_b}]{\includegraphics[width=0.3\textwidth]{chapters/chapter2_theory/images/SingleResonant.pdf}}
		\subfloat[\label{fig:hpm-diagrams_c}]{\includegraphics[width=0.3\textwidth]{chapters/chapter2_theory/images/DoubleResonant.pdf}}
		\caption{\label{fig:hpm-diagrams} Examples of leading-order Feynman diagrams contributing to the production of charged Higgs bosons in $pp$ collisions: (a) non-resonant top-quark production prevalent in the intermediate-mass range, (b) single-resonant top-quark production that dominates at large \Hpm masses, (c) double-resonant top-quark production that dominates at low \Hpm masses. The interference between these three diagrams becomes most relevant in the intermediate-mass region.}
	\end{figure}
	\begin{figure}[!ht]
		\centering
		\includegraphics[width=0.75\textwidth]{chapters/chapter2_theory/images/XSBR_hmssm.pdf}
		\caption{\label{fig:hpm-xsec} Variation of $\sigma(\pp \to tb\Hpm) \times B(\HpmLong)$ with the charged Higgs boson mass in \pp collisions at \sqs, for \tanb values of 7, 20, and 40 in the hMSSM scenario. Dashed lines correspond to $B(\HpmLong)$ set to 1, hence they show the dependence of $\sigma(\pp \to tb\Hpm)$ with \mHpm. Taken from Reference \cite{hpm-previous}.}
	\end{figure}
	The production diagrams considered in this dissertation can be seen in Figure \ref{fig:hpm-diagrams}. The cross section at various \tanb values can be seen as a function of \mHpm in Figure \ref{fig:hpm-xsec}. The cross section scales with \tanb and at very small values the top Yukawa couplings become non-perturbative, meaning they are very difficult to predict and very unlikely to occur. In this dissertation the decay channel considered is \HpmLong. 
	\begin{figure}[!ht]
		\centering
		\subfloat[\label{fig:hpm-br_a}]{\includegraphics[width=0.5\textwidth]{chapters/chapter2_theory/images/YRHXS3_BR_fig33.eps}}
		\subfloat[\label{fig:hpm-br_b}]{\includegraphics[width=0.5\textwidth]{chapters/chapter2_theory/images/YRHXS3_BR_fig34.eps}}
		\caption{\label{fig:hpm-br} Branching fractions of \Hpm as a function of \mHpm for (a) $\tanb = 10$ and (b) $\tanb = 50$ in the $m^{mod+}_{h}$ scenario of the \gls{MSSM} \cite{Higgs-Crosssections}. }
	\end{figure}
	As can be seen in Figure \ref{fig:hpm-br}, the \HpmLong decay channel is especially relevant at low \mHpm and high \tanb. This dissertation describes a search for charged Higgs bosons produced in association with a top quark, where only the \HpmLong decay channel is considered. Other decay modes of the \Hpm to \gls{SM} particles are covered in other searches \cites{MSSM-benchmarks}. Decay channels of \Hpm to other \gls{MSSM} are not considered in this dissertation. The search consists of two sub-channels, \taujets and \taulep, where the associated top quark decays either hadronically or leptonically respectively. 

	Within the \gls{MSSM}, several benchmarks are defined taking into account higher-order corrections and keeping the number of free parameters in the model small \cite{MSSM-benchmarks}. Figure \ref{fig:hpm-xsec} is made assuming the hMSSM model, where $h^0$ is taken as the observed 125 GeV Higgs and the absence of observed \gls{SUSY} at the LHC is taken into account by setting the \gls{SUSY} scale to $M_{SUSY}>1$ TeV \cite{hMSSM}. Figure \ref{fig:hpm-br} shows the branching ratios of \Hpm for various \tanb values in the $m^{mod+}_{h}$ model where the benchmark scenario $m^{max}_{h}$ has been modified to interpret $h$ as the observed boson\footnote{The $m^{max}_{h}$ scenario is constructed to yield the highest possible mass for h at any given \tanb.} \cite{MSSM-benchmarks}.

	\subsection{Previous Result}\label{ssec:Prev Hpm}
		To add context to this dissertation, it is important to reference the results of the previous iteration of the search discussed in this dissertation\footnote{The author joined the analysis team towards the end of this iteration and performed validation studies.}. The \gls{ATLAS} collaboration published a paper in 2018 covering the data taking years of 2015 and 2016 \cite{hpm-previous}, whereas this dissertation covers the full Run-2 (2015-2018) dataset. 
		\begin{figure}[!ht]
			\centering
			\subfloat[\label{fig:hpm-prev-limits_a}]{\includegraphics[width=0.5\textwidth]{chapters/chapter2_theory/images/Previous_Limits_Taujets.png}}
			\subfloat[\label{fig:hpm-prev-limits_b}]{\includegraphics[width=0.5\textwidth]{chapters/chapter2_theory/images/Previous_Limits_Taulep.png}} \\
			\subfloat[\label{fig:hpm-prev-limits_c}]{\includegraphics[width=0.75\textwidth]{chapters/chapter2_theory/images/Previous_Limits_Combined.png}}
			\caption{\label{fig:hpm-prev-limits} Exclusion limits at 95\% CL on $\sigma(pp \to tbH^{\pm} \times B(H^{\pm} \to \tau^\pm \nu)) $ [pb]. Top left (a) shows the \taujets subchannel, top right (b) corresponds to the \taulep subchannel and bottom (c) shows the combination of the two subchannels. Taken from Reference \cite{hpm-previous}. }
		\end{figure}
		\begin{figure}[!ht]
			\centering
			\includegraphics[width=0.75\textwidth]{chapters/chapter2_theory/images/Previous_Limits_Combined_tanb.png}
			\caption{\label{fig:hpm-prev-limits-tanb} Exclusion limits at 95\% CL on \tanb as a function of \mHpm \cite{hpm-previous}. }
		\end{figure}
		Figure \ref{fig:hpm-prev-limits} shows the limits on the cross section and Figure \ref{fig:hpm-prev-limits-tanb} shows the limits on \tanb as a function of \mHpm. 

		There are many other searches for charged Higgs bosons that have been performed by several experiments in many different decay channels. Figure \ref{fig:hpm-taunu-cms-limits} shows the results from CMS in the same \HpmLong decay channel \cite{CMS-taunu}. Figure \ref{fig:hpm-tb-limits} shows the ATLAS results for a charged Higgs search in the $\Hpm \to tb$ decay channel and Figure \ref{fig:hpm-tb-tanb-limits} shows the previous \HpmLong search results overlayed on the $\Hpm \to tb$ results in two theoretical models \cite{Hpm-to-tb}.

		\begin{figure}[!ht]
			\centering
			\subfloat[\label{fig:hpm-taunu-cms-limits_a}]{\includegraphics[width=0.5\textwidth,keepaspectratio=true]{chapters/chapter2_theory/images/CMS_taunu_xsec_limits.png}}
			\subfloat[\label{fig:hpm-taunu-cms-limits_b}]{\includegraphics[width=0.5\textwidth,keepaspectratio=true]{chapters/chapter2_theory/images/CMS_taunu_tanb_limits.png}}
			\caption{\label{fig:hpm-taunu-cms-limits} (a) Exclusion limits at 95\% CL on $\sigma_{\Hpm} B(\Hpm \to \tau^\pm \nu_{\tau})$ [pb] from the CMS collaboration \cite{CMS-taunu}. (b) Exclusion limits at 95\% CL on \tanb and \mHpm \cite{CMS-taunu}. }
		\end{figure}

		\begin{figure}[!ht]
			\centering
			\includegraphics[width=0.75\textwidth]{chapters/chapter2_theory/images/HPlus_to_tb_Limits.png}
			\caption{\label{fig:hpm-tb-limits} Exclusion limits at 95\% CL on $\sigma (pp \to tb\Hpm) \times B(\Hpm \to tb)$ as a function of \mHpm \cite{Hpm-to-tb}. }
		\end{figure}

		\begin{figure}[!ht]
			\centering
			\subfloat[\label{fig:hpm-tb-tanb-limits_a}]{\includegraphics[width=0.5\textwidth,keepaspectratio=true]{defense/HPlus_taunu_tb_tanb_Limits_hMSSM.png}}
			\subfloat[\label{fig:hpm-tb-tanb-limits_b}]{\includegraphics[width=0.5\textwidth,keepaspectratio=true]{defense/HPlus_taunu_tb_tanb_Limits_hMod.png}}
			\caption{\label{fig:hpm-tb-tanb-limits} Exclusion limits at 95\% CL on \tanb as a function of \mHpm in the (a) \gls{MSSM} and (b) $m^{mod+}_{h}$ scenario of the \gls{MSSM} \cite{Hpm-to-tb}. }
		\end{figure}


		The previous iteration of this analysis relied on boosted decision trees (BDT) binned in \mHpm, using five separate classifiers to cover the mass range of $90 - 2000$ GeV. The mass bins can be seen as the blue dotted lines in Figure \ref{fig:hpm-prev-limits_c}. It is important to note the inclusion of the mass range $90 - 200$ GeV, as Reference \cite{hpm-previous} was the first search to include this mass region below the top quark mass.